%%% The main file. It contains definitions of basic parameters and includes all other parts.

%% Settings for single-side (simplex) printing
% Margins: left 40mm, right 25mm, top and bottom 25mm
% (but beware, LaTeX adds 1in implicitly)
\documentclass[12pt,a4paper]{report}
\setlength\textwidth{145mm}
\setlength\textheight{247mm}
\setlength\oddsidemargin{15mm}
\setlength\evensidemargin{15mm}
\setlength\topmargin{0mm}
\setlength\headsep{0mm}
\setlength\headheight{0mm}
% \openright makes the following text appear on a right-hand page
\let\openright=\clearpage

%% Settings for two-sided (duplex) printing
% \documentclass[12pt,a4paper,twoside,openright]{report}
% \setlength\textwidth{145mm}
% \setlength\textheight{247mm}
% \setlength\oddsidemargin{14.2mm}
% \setlength\evensidemargin{0mm}
% \setlength\topmargin{0mm}
% \setlength\headsep{0mm}
% \setlength\headheight{0mm}
% \let\openright=\cleardoublepage

%% Character encoding: usually latin2, cp1250 or utf8:
\usepackage[utf8]{inputenc}

%% Further useful packages (included in most LaTeX distributions)
\usepackage{amsmath}        % extensions for typesetting of math
\usepackage{amsfonts}       % math fonts
\usepackage{amsthm}         % theorems, definitions, etc.
\usepackage{bbding}         % various symbols (squares, asterisks, scissors, ...)
\usepackage{bm}             % boldface symbols (\bm)
\usepackage{graphicx}       % embedding of pictures
\usepackage{fancyvrb}       % improved verbatim environment
\usepackage{natbib}         % citation style AUTHOR (YEAR), or AUTHOR [NUMBER]
\usepackage[nottoc]{tocbibind} % makes sure that bibliography and the lists
			    % of figures/tables are included in the table
			    % of contents
\usepackage{dcolumn}        % improved alignment of table columns
\usepackage{booktabs}       % improved horizontal lines in tables
\usepackage{paralist}       % improved enumerate and itemize
\usepackage[usenames]{xcolor}  % typesetting in color
\usepackage{listings}
\usepackage{caption}
\usepackage{algorithm}
\usepackage[noend]{algpseudocode}
\usepackage{csquotes}
\usepackage{url}


%%% Basic information on the thesis

% Thesis title in English (exactly as in the formal assignment)
\def\ThesisTitle{Performance based adaptation of Scala programs}

% Author of the thesis
\def\ThesisAuthor{Petr Kubát}

% Year when the thesis is submitted
\def\YearSubmitted{2017}

% Name of the department or institute, where the work was officially assigned
% (according to the Organizational Structure of MFF UK in English,
% or a full name of a department outside MFF)
\def\Department{Department of Distributed and Dependable Systems}

% Is it a department (katedra), or an institute (ústav)?
\def\DeptType{Department}

% Thesis supervisor: name, surname and titles
\def\Supervisor{Tomáš Bureš}

% Supervisor's department (again according to Organizational structure of MFF)
\def\SupervisorsDepartment{Department of Distributed and Dependable Systems}

% Study programme and specialization
\def\StudyProgramme{Software Systems}
\def\StudyBranch{Software Engineering}

% An optional dedication: you can thank whomever you wish (your supervisor,
% consultant, a person who lent the software, etc.)
\def\Dedication{%
I would like to thank my supervisor, Doc. RNDr. Tomáš Bureš, PhD., for his help, valuable suggestions and feedback. I would also like to thank prof. Ing. Petr Tůma, Dr. for his consultations and advice. Last but not least, I would like to thank Mgr. Vojtěch Horký for setting up the environment to perform Spark tests.
}

% Abstract (recommended length around 80-200 words; this is not a copy of your thesis assignment!)
\def\Abstract{%
Abstract.
}

% 3 to 5 keywords (recommended), each enclosed in curly braces
\def\Keywords{%
{key} {words}
}

%% The hyperref package for clickable links in PDF and also for storing
%% metadata to PDF (including the table of contents).
\usepackage[pdftex,unicode]{hyperref}   % Must follow all other packages
\hypersetup{breaklinks=true}
\hypersetup{pdftitle={\ThesisTitle}}
\hypersetup{pdfauthor={\ThesisAuthor}}
\hypersetup{pdfkeywords=\Keywords}
\hypersetup{urlcolor=blue}

% Definitions of macros (see description inside)
%%% This file contains definitions of various useful macros and environments %%%
%%% Please add more macros here instead of cluttering other files with them. %%%

%%% Minor tweaks of style

% These macros employ a little dirty trick to convince LaTeX to typeset
% chapter headings sanely, without lots of empty space above them.
% Feel free to ignore.
\makeatletter
\def\@makechapterhead#1{
	{\parindent \z@ \raggedright \normalfont
		\Huge\bfseries \thechapter. #1
		\par\nobreak
		\vskip 20\p@
}}
\def\@makeschapterhead#1{
	{\parindent \z@ \raggedright \normalfont
		\Huge\bfseries #1
		\par\nobreak
		\vskip 20\p@
}}
\makeatother

% This macro defines a chapter, which is not numbered, but is included
% in the table of contents.
\def\chapwithtoc#1{
	\chapter*{#1}
	\addcontentsline{toc}{chapter}{#1}
}

% Draw black "slugs" whenever a line overflows, so that we can spot it easily.
\overfullrule=1mm

%%% Macros for definitions, theorems, claims, examples, ... (requires amsthm package)

\theoremstyle{plain}
\newtheorem{thm}{Theorem}
\newtheorem{lemma}[thm]{Lemma}
\newtheorem{claim}[thm]{Claim}

\theoremstyle{plain}
\newtheorem{defn}{Definition}

\theoremstyle{remark}
\newtheorem*{cor}{Corollary}
\newtheorem*{rem}{Remark}
\newtheorem*{example}{Example}

%%% An environment for proofs

%%% FIXME %%% \newenvironment{proof}{
%%% FIXME %%%   \par\medskip\noindent
%%% FIXME %%%   \textit{Proof}.
%%% FIXME %%% }{
%%% FIXME %%% \newline
%%% FIXME %%% \rightline{$\square$}  % or \SquareCastShadowBottomRight from bbding package
%%% FIXME %%% }

%%% An environment for typesetting of program code and input/output
%%% of programs. (Requires the fancyvrb package -- fancy verbatim.)

\DefineVerbatimEnvironment{code}{Verbatim}{fontsize=\small, frame=single}

%%% The field of all real and natural numbers
\newcommand{\R}{\mathbb{R}}
\newcommand{\N}{\mathbb{N}}

%%% Useful operators for statistics and probability
\DeclareMathOperator{\pr}{\textsf{P}}
\DeclareMathOperator{\E}{\textsf{E}\,}
\DeclareMathOperator{\var}{\textrm{var}}
\DeclareMathOperator{\sd}{\textrm{sd}}

%%% Transposition of a vector/matrix
\newcommand{\T}[1]{#1^\top}

%%% Various math goodies
\newcommand{\goto}{\rightarrow}
\newcommand{\gotop}{\stackrel{P}{\longrightarrow}}
\newcommand{\maon}[1]{o(n^{#1})}
\newcommand{\abs}[1]{\left|{#1}\right|}
\newcommand{\dint}{\int_0^\tau\!\!\int_0^\tau}
\newcommand{\isqr}[1]{\frac{1}{\sqrt{#1}}}

%%% Various table goodies
\newcommand{\pulrad}[1]{\raisebox{1.5ex}[0pt]{#1}}
\newcommand{\mc}[1]{\multicolumn{1}{c}{#1}}

% Settings of the code listings
\usepackage{listings}
\usepackage{color}
\usepackage[T1]{fontenc}
\usepackage[scaled]{beramono}

\usepackage{color}
%\definecolor{bluekeywords}{rgb}{0.13,0.13,1}
\definecolor{bluekeywords}{rgb}{0,0,0.5}
\definecolor{greencomments}{rgb}{0,0.5,0}
\definecolor{redstrings}{rgb}{0.9,0,0}
\definecolor{mygray}{rgb}{0.5,0.5,0.5}

\lstdefinestyle{Scala}{  
	language=Scala,
	showspaces=false,
	showtabs=false,
	breaklines=true,
	showstringspaces=false,
	breakatwhitespace=true,
	columns=fullflexible,
	escapeinside={(*@}{@*)},
	otherkeywords={},
	commentstyle=\color{greencomments},
	keywordstyle=\color{bluekeywords}\bfseries,
	stringstyle=\color{redstrings},
	basicstyle=\ttfamily,
	numbers=left,                    % where to put the line-numbers; possible values are (none, left, right)
	numbersep=12pt,                   % how far the line-numbers are from the code
	numberstyle=\color{mygray}, % the style that is used for the line-numbers
	rulecolor=\color{black},         % if not set, the frame-color may be changed on line-breaks within not-black text (e.g. comments (green here))
	stepnumber=1,                    % the step between two line-numbers. If it's 1, each line will be numbered
}

\lstset{
	language=Scala,
	showspaces=false,
	showtabs=false,
	breaklines=true,
	showstringspaces=false,
	breakatwhitespace=true,
	columns=fullflexible,
	escapeinside={(*@}{@*)},
	otherkeywords={},
	commentstyle=\color{greencomments},
	keywordstyle=\color{bluekeywords}\bfseries,
	stringstyle=\color{redstrings},
	basicstyle=\ttfamily,
	numbers=left,                    % where to put the line-numbers; possible values are (none, left, right)
	numbersep=12pt,                   % how far the line-numbers are from the code
	numberstyle=\color{mygray}, % the style that is used for the line-numbers
	rulecolor=\color{black},         % if not set, the frame-color may be changed on line-breaks within not-black text (e.g. comments (green here))
	stepnumber=1,                    % the step between two line-numbers. If it's 1, each line will be numbered
}

% Title page and various mandatory informational pages
\begin{document}
\include{title}

%%% A page with automatically generated table of contents of the master thesis

\tableofcontents

%%% Each chapter is kept in a separate file
\chapter*{Introduction}
\addcontentsline{toc}{chapter}{Introduction}

In modern software engineering, performance awareness has become a first class concern. The systems deal with growing amounts of data, with higher emphasis on the real-time processing and with more limited environments for embedded solutions, while we are reaching hardware limitations. It is the responsibility of the developer to assure that his code will not only work correctly, but will also be optimized and fast enough to meet these expectations.

The actual performance-aware programming can often be difficult and problematic for the developer, who is forced to make an assumptions about the speed of his program. The performance in general is, however, a dynamic and not a static feature. Even though it is possible to analyze the algorithm complexity at design-time, the result will always be only an approximation, as it works with abstract elementary computational steps. In reality, the programs often behave in a different way due to the influence of CPU\footnote{Central Processing Unit} cache, memory locality, I/O\footnote{Input / Output operation} operations, parallelism and many other factors. 

The only way how to actually find out if given implementation is fast enough - possibly faster than some other variant - is through a thorough testing process covering all the possible inputs and executed in the actual production environment. This process is complicated, costly, and should be repeated after every change in the code, as it can have a previously unexpected impact. It can be partially automatized by introducing performance constraints to the classic unit tests. There is an active research in this area, e.g. \cite{bulej_capturing_2012,horky_performance_2013,horky_utilizing_2015}.

Considering a common case where a programmer needs to decide which implementation option of some functionality to use, the discussed solution has some major downsides. First, he needs to design the tests, select test data, execute them and make decisions based on the results, which is a non-negligible effort, especially in a common software engineering process. Second, the decision that is made is final - once executed, the program will always have to use only that selected implementation. There are many cases where the performance of the implementations varies for different inputs or in different environments, and we would like our application to always use the best possible implementation, in other words, to adapt its execution to these conditions.

To address these problems and to simplify the performance-aware development in general, we propose a completely new approach in development, where the programmer identifies the implementation options of a function in the code using a programming language construct. At execution, the system tracks the performance of the implementations involved and makes a new decision for each run of the function based on the inputs and current trends in the run times.

The goal of this thesis is to design a framework that would allow this way of development and to implement its prototype. It will be tested in various scenarios common for many applications where the approach might be beneficial, and the actual advantages and problems of proposed solution will be evaluated. The target platform of both the design and the implementation is the Scala programming language (\cite{noauthor_scala_nodate}), because it has a very flexible syntax suitable for creating language-like constructs while being statically typed with both object-oriented and functional foundations. In addition, it compiles to the Java bytecode and runs on the JVM\footnote{Java Virtual Machine}, which makes it portable and open to the variety of \mbox{Java-based} libraries. 

Many frameworks for data processing and similar tasks where the adaptation could be employed are either implemented or have interfaces in Scala, mainly for its high expressive powers. An example can be the Spark framework \cite{noauthor_apache_nodate}, which will be used in the evaluation process.

\subsubsection{Structure of the text}

In the first chapter of this work, the Scala language features that are not common among current object-oriented programming languages are introduced. Most of them will be mentioned and used in further chapters, so it serves as a brief introduction for a reader who knows basic object-oriented and functional languages, but does not have a deep knowledge of advanced Scala features.

The second chapter serves as a brief overview of the framework developed as a part of this thesis. More detailed goals are stated and a basic structure of the solution is outlined.

The third chapter is dedicated to the framework API. It begins with the requirement analysis and most simple use cases, followed up by possible drafts and their problems. Then, the actual design is presented, along with its evaluation and more advanced scenarios and extensions. The whole chapter is written independently on the framework implementation, and is based only on the requirements and a basic notion of the functionality.

The fourth chapter is more theoretically focused. It introduces the functionality of the selection part of the framework on an abstract level. The whole process of deciding between multiple functions based on the historical observations of their runs is explained. Several algorithms are presented, tested and compared. In addition, more concrete improvements are suggested to increase the selection precision and to speed up the invocation process.

In the fifth chapter, the whole framework implementation is briefly presented. Concepts from the third and fourth chapters are incorporated into an actual Scala program. A couple of problems is mentioned and solved. Last but not least, the options that the framework user has of customizing and possibly extending the functionality are discussed.

The goal of the sixth chapter is to try out the framework in real-life scenarios and to evaluate the benefits that it brings. For that, a couple of different problems will be used, including basic selection between algorithms, JSON parsing or adapting to quickly changing network environment. A part of the chapter is devoted to the Spark framework for distributed data processing, which offers a great potential for adaptive execution.

In the last, seventh chapter, other work that is related to the problem of performance adaptation, measurement, prediction and similar topics is analyzed.
\chapter{Scala programming language}

\section{Important language features}

%TODO: scala implicit casts

%TODO: scala function types (and nececity to duplicate code)

%TODO: extending function type, overriding apply()

\subsection{Methods and functions in Scala}
\label{subsec:metandfun}

There is an important difference between methods and functions in Scala. Methods have the same concept as in Java or other object oriented languages - they are an integral, compile-time part of their class and the only context that they can refer to is their class instance, arguments or static members. They can't be passed around, so no additional context is needed. Every invocation has to be performed on a class instance, allowing the compiler to correctly assign the \textit{this} reference. Simple example of a method can be the following:

\lstset{language=Scala}
\begin{lstlisting}
class Class {
def method(arg: String): String = s"Called method($arg) on $this"
}
val obj = new Class()
println(obj.method("Hello"))
\end{lstlisting}

On the other hand, functions are first-class values that have a type and can be passed into other functions or methods, or be part of expressions. Internally, they are represented as closure objects, carrying around their context and the compiled code stored as a method in the closure type. Functions are created using lambda expressions:

\lstset{language=Scala}
\begin{lstlisting}
val function: (String) => Unit = arg => println(obj.method(arg))
function("Hello again!")
List("One", "Two", "Three").foreach(function)
\end{lstlisting}

The \lstinline|obj| reference is \textit{captured} inside the closure, the function is not tied to any class. 



\subsection{Eta-expansion}
\label{subsec:etaexpansion}

Because the strengths of Scala lie in its functional features, it is often handy to be able to convert methods to functions. This is called \textit{eta-expansion} and is quite straightforward (assuming we are inside the class):

\lstset{language=Scala}
\begin{lstlisting}
val methodFunction: (String) => String = arg => method(arg)
\end{lstlisting}

Scala has a special operator that makes it even simpler:

\lstset{language=Scala}
\begin{lstlisting}
val methodFunction = method _
\end{lstlisting}

And in some expression contexts, this conversion is implicit:

\lstset{language=Scala}
\begin{lstlisting}
List("One", "Two", "Three").map(method)
\end{lstlisting}

In this example, the conversion is implicitly applied to an argument of a method in order to meet the parameter type. The expansion in this case will be always implicit, unless one of the following conditions is met:
\begin{enumerate}
	\item The method has overloads.
	\item There is an implicit typecast required from the expanded type to the parameter type.
\end{enumerate}

%TODO: Generic methods vs. functions

%TODO: methods vs. functions in scala, eta expanstion

%TODO: multiple combination

%TODO: covariance and contravariance

\subsection{Function types}
\label{subsec:functiontypes}

The first-class value functions have to be anchored somewhere in the Scala type system in order for the type inference, type checking and other mechanisms to work correctly. The type of a function is unambiguously determined by its signature, i.e. the number and types of arguments and the type of the return type. Consequentially, the function types have to be generic and able to accept different numbers of type arguments.

Scala as a language doesn't support variadic generic types (as opposed to languages like C++), so there has to be a different type with different number of type arguments for every possible number of arguments of a function. Currently, there are traits for functions accepting 0 to 22 arguments:

\lstset{language=Scala}
\begin{lstlisting}
trait Function0[+R]
...
trait Function22[-T1, ..., -T22, +R]
\end{lstlisting}

These traits have syntactic aliases in the language:

\lstset{language=Scala}
\begin{lstlisting}
() => R
...
(T1, ..., T22) => R
\end{lstlisting}

Functions with more than 22 arguments can't exist in Scala and have to be replaced by function accepting tupled arguments.

The necessity to represent functions with various argument numbers as various traits leads to code duplication.

\section{DSLs in Scala}

\subsection{Infix operators}
\label{subsec:infixops}

One of the features that support both the functional flavor of Scala and the DSL building capabilities is the possibility to call methods like infix operators. Any method that has exactly one argument can be called using this special syntax:

\lstset{language=Scala}
\begin{lstlisting}
class RichInt {
  def plus(other: RichInt) = ???
}
...
val res = x plus y
\end{lstlisting}

The syntax is even more powerful if we take into consideration that Scala allow the methods to have non-alphanumeric names:

\lstset{language=Scala}
\begin{lstlisting}
def +(other: RichInt) = ???
...
val res = x + y
\end{lstlisting}

This approach allows us to use the infix syntax with existing methods and with methods that weren't designed with the intention to be used that way. It has, however, a downside as well - it doesn't allow us to create an infix operator that would accept as its first argument a type without actually changing the type and adding the method to its definition. Scala doesn't directly support declaring extension methods on types (unlike C\#, Kotlin and other languages) and so the only way to solve this problem is to introduce implicit typecast to a custom type and to add the infix operator as a method to it.

% TODO: Compare to infix operators in Kotlin

\section{Def Macros}

\lstset{language=Scala}
\begin{lstlisting}
def printAst(arg: Any): Any = macro printAst_impl

def printAst_impl(c: Context)(arg: c.Expr[Any]): c.Expr[Any] = {
  import c.universe._
  println(showRaw(arg.tree))
  arg
}
\end{lstlisting}
\chapter{Solution overview}

This chapter will provide a basic overview of the solution presented and of the tasks that had to be solved in the process and that will be described with more details in later sections.

%TODO: Maybe add references to future chapters

\section{Detailed goals}
\label{sec:goals_revisited}

In the introduction, the basic goal of the thesis was mentioned in a very general manner - as a simple-to-use framework that allows the user to introduce variants into his program. Now, we will go a little deeper and present more detailed goals and the consequences they have for the implementation.

\subsubsection{A simple and fluent API}

We would like the programmer to be able to simply combine two or more functions or methods (in any combination), resulting into a new function that can be called again.

\lstset{style=Scala}
\begin{lstlisting}
def impl1(in: T): U = ???
def impl2(in: T): U = ???
val impl3: (T) => U = ???

val function = impl1 _ or impl2 or impl3
\end{lstlisting}

This means that we are going to have to work with implicit typecasts from function types and with eta-expansion.

\subsubsection{Transparent usage}

The function that was created by the combining process (in the rest of the text referred to as \textit{combined function}) should behave just like a normal function - the caller should not have to perform any special action and he should not be able to recognize that multiple implementations are involved.

\lstset{style=Scala}
\begin{lstlisting}
val result1: U = impl1(arg)
val result2: U = function(arg)
\end{lstlisting}

If the user, however, decides to combine the function that is already combined from $n$ simple functions one more time, the result should be a combination of $n+1$ functions, not a combination of $2$ functions where one of them is the original.

\lstset{style=Scala}
\begin{lstlisting}
val newFunction = function or impl4
\end{lstlisting}

Similar behavior can be achieved by introducing custom function trait implementations and reflecting it in the typecast.

\subsubsection{Measuring and storing run times}

Whenever the function is called and one of the implementations run, the run time (or other metric) has to be measured and stored. The storage should be statically accessible, because the function can be invoked from different contexts and we would like to share the history across the entire application. In order to identify the implementation in the static storage, we would need an identifier, preferably the name of the method (if a method is used). For that, we are going to have to use a compile-time macro.

\subsubsection{Selecting the most appropriate function in different cases}

As the key part of the library, we want to select the function to run each time. The selection should have 2 basic outcomes:

\begin{itemize}
	\item Select the function that is expected to have better performance on given input in current conditions if we are certain enough
	\item Select the function that we need to collect data for if we are not certain enough
\end{itemize}

The selection would be based on historical measurements of each function. We should be able to handle the following situations regarding the history data:

\begin{itemize}
	\item Functions whose run time does not depend significantly on the input - we expect that if one function has better performance than the other, it will not change with different input or environment
	\item Functions whose run time depends on some characteristics of the input (e.g. if the sequence is sorted)
	\item Functions whose run time is expected to be a function of the input size or some similar measurable feature of the input (e.g. length of the sequence to sort)
	\item Functions whose run time depends on the execution environment and is expected to change over time
\end{itemize}

This leads to a few requirements. Multiple selection strategies meeting the requirements of these cases will be necessary - we need to propose some decision making procedures that will reflect the certainty using the methods of statistical testing with given significance level. The user will have a way to customize the function combination in order to identify the case and configure the selection process correctly.

\subsubsection{Controlling the invocation behavior}

The invocation behavior should change in some situations - when we are sure enough that one option is better than the other in general, when we need to gather more data, when we need to save time and cannot perform the selection, or in other cases. Every combined function should have its state and some behavior plan, which would reflect some basic statistics and events on the functions.

We can achieve this by introducing a simple state machine logic and a language in form of a Scala DSL to create simple state machines.

\subsubsection{Possibility to extend the framework}

The framework should be modular, with its parts being easy to replace or extend. It should be possible for the user of the framework to incorporate his own implementations without having to change the framework code in any way, just by modifying the initial runtime construction.

The following extensions should be possible without changing the code of the framework, only by adding custom implementations:

\begin{itemize}
	\item Change the metrics that is measured for the functions
	\item Change the history storage
	\item Add, replace or extend the selection strategies
\end{itemize}

For this to work, the framework runtime will be created by composition with no close-coupling between modules. All of the runtime parts will communicate using trait interfaces. A simple form of IoC\footnote{Inversion of Control} configuration will be used for the composition.

\section{Key parts of the system}

Based on the extended goals presented in section \ref{sec:goals_revisited}, key parts of the solution can be identified and will be described in detail in the rest of the text.

\begin{itemize}
	\item \textbf{API and \inlinecode{CombinedFunction}} - a set of classes and methods that the user will be directly interacting with in order to and invoke the combined functions
	\item \textbf{Invocation policies} - a system of simple state machines to manage the invocation process with almost no overhead, handled directly in the \textit{CombinedFunction}
	\item \textbf{Selection strategies} - simple, stateless and isolated strategies that can decide about the most appropriate function to run given a historical measurements of all the functions
	\item \textbf{Function selector and invoker with the history storage} - module that will be responsible for selecting the method using a given strategy, invoking it, measuring the performance metrics and storing and retrieving the history data; it will be called by the \textit{CombinedFunction}
	\item \textbf{Configuration} - a static description of the function selector and invoker, can be changed by the user
\end{itemize}
\chapter{Adaptation API design and implementation}

\section{The importance of API design}

API (Application Programming Interface) is a key component of every framework or library. It defines how a stand-alone piece of code (a function, class, module, or even entire running application) interacts with its environment. This includes the possible calls / requests, format of the data that is passed in as arguments and the data that is received as a result of the action.

APIs can be found on multiple levels of abstraction in a specific piece of software - % TODO 

.In our case, the API will be used to access functionality of our framework after adding our classes to the project.

\section{Basic API requirements}

The library itself should not require much interaction from the programmer. The usage would be based on performing some initial configuration, marking method implementations that are interchangeable, and then calling repeatedly either one of the methods marked, or some special method, and receiving correct results from one of the implementations.
In the optimal case, the library would be just added as a reference into the project and some minor changes would be done at the highest level, i.e. in the class method definitions, traits, adapters, etc. The business logic of the application should remain intact.
We can make a list of the \textbf{basic requirements}:
\begin{enumerate}
	\item Mark two or more methods as linked together (stating that they can be called interchangeably)
	\item Perform a call to the group of linked methods
	\item %TODO
\end{enumerate}

And some possible \textbf{extensions} to the API:
\begin{enumerate}
	\item Separate the selecting call and the evaluation call
	\item %TODO
\end{enumerate}

\section{Goals of the API design}

For a framework that will be used repeatedly by a variety of other developers, the API design is crucial and has several principal goals:

\begin{enumerate}
	\item Keeping the API calls simple in simple cases
	\item Giving the caller more options in more complicated cases
\end{enumerate}

\section{Advantages and disadvantages of Scala in API design}

% TODO: Macros

\section{Possibe API drafts}

\subsection{Direct interaction}

\subsection{Function composition}

Scala is a programming language that has absorbed a lot of concepts from the functional programming language world. Above all the key concept that functions are \textit{first-class values} and can be passed around the same way as code.

The goal of our API is to allow the programmer to link together two or more functions and receive one universal function that will decide which one to call. In a purely functional language, this could be solved using a \textit{higher-order function}\footnote{A function whose arguments and return type are other functions.}. It is a common approach

\lstset{language=Haskell}
\begin{lstlisting}
funA :: Ord a => [a] -> [a]
funB :: Ord a => [a] -> [a]

fun = selectFrom funA funB

fun :: Ord a => [a] -> [a]
\end{lstlisting}

\section{Scala API implementation}

%TODO: scala implicit casts

%TODO: scala function types (and nececity to duplicate code)

%TODO: extending function type, overriding apply()

\subsection{Methods and functions in Scala}

There is an important difference between methods and functions in Scala. Methods have the same concept as in Java or other object oriented languages - they are an integral, compile-time part of their class and the only context that they can refer to is their class instance, arguments or static members. They can't be passed around, so no additional context is needed. Every invocation has to be performed on a class instance, allowing the compiler to correctly assign the \textit{this} reference. Simple example of a method can be the following:

\begin{lstlisting}
class Class {
def method(arg: String): String = s"Called method($arg) on $this"
}
val obj = new Class()
println(obj.method("Hello"))
\end{lstlisting}

On the other hand, functions are first-class values that have a type and can be passed into other functions or methods, or be part of expressions. Internally, they are represented as closure objects, carrying around their context and the compiled code stored as a method in the closure type. Functions are created using lambda expressions:

\begin{lstlisting}
val function: (String) => Unit = arg => println(obj.method(arg))
function("Hello again!")
List("One", "Two", "Three").foreach(function)
\end{lstlisting}

The \lstinline|obj| reference is \textit{captured} inside the closure, the function is not tied to any class. 

Because the strengths of Scala lie in its functional features, it is often handy to be able to convert methods to functions. This is called \textit{eta-expansion} and is quite straightforward (assuming we are inside the class):

\begin{lstlisting}
val methodFunction: (String) => String = arg => method(arg)
\end{lstlisting}

Scala has a special operator that makes it even simpler:

\begin{lstlisting}
val methodFunction = method _
\end{lstlisting}

And in some expression contexts, this conversion is implicit:

\begin{lstlisting}
List("One", "Two", "Three").map(method)
\end{lstlisting}

%TODO: Generic methods vs. functions

%TODO: methods vs. functions in scala, eta expanstion

%TODO: multiple combination

%TODO: covariance and contravariance

\subsection{Covariance and contravariance}
%TODO: Object-oriented polymorphism
% http://milessabin.com/blog/2012/04/27/shapeless-polymorphic-function-values-1/

So far, all mentioned usages of the \lstinline|or()| method were limited to functions with the same signatures. Quite common case, however, might be combining multiple functions with slightly different argument and return value types, typically one being a specialized version of the other. For example:

%TODO: replace this example:

The two functions can't be joined using the \lstinline|or()| method, because \lstinline|bubbleSort()| is defined on a more general type than \lstinline|radixSort()|. Let's examine simplified case with more combinations:

\lstset{language=Scala}
\begin{lstlisting}
def fun1(arg: Any): String = ???
def fun2(arg: String): Any = ???
def fun3(arg: String): String = ???

val fun4 = fun3 _ or fun2
val fun5 = fun2 _ or fun3
val fun6 = fun3 _ or fun1
val fun7 = fun1 _ or fun3
\end{lstlisting}

We receive compilation error on lines 5 and 8, in the definition of \lstinline|fun4| and \lstinline|fun7|. Lines 6 and 7 compile correctly. These are the cases where:

\begin{enumerate}
	\item Return type of the function passed in as an argument is a subtype of the return type of the target function
	\item Argument type of the function passed in as an argument is a supertype of the argument type of the target function
\end{enumerate}

The function types (represented as traits) in Scala are defined in the following way:
\lstset{language=Scala}
\begin{lstlisting}
trait Function1[-T1, +R] extends AnyRef
\end{lstlisting}

The type arguments representing the function arguments are defined as contravariant and the type argument representing the return value of the function is defined as covariant. This leads to \lstinline|(String) => String| being a subtype of \lstinline|(String) => Any| and to \lstinline|(Any) => String| being a subtype of \lstinline|(String) => String|. So the \lstinline|or()| method is working flawlessly for the \lstinline|fun5| and \lstinline|fun6|.

Taking into account that \lstinline|or()| is behaving like a commutative infix operator, we would like it to work in the other two cases as well. The two functions combined will always be in a subtype - supertype relation The signature of the function returned has to match the more limiting signature, i.e. the signature,

\begin{lstlisting}
def funA(arg: A): A = ???
def funB(arg: B): B = ???
val funAB: (B with A) => Object = funA _ or funB
\end{lstlisting}

%TODO: Comment case & listing

As we can see, these are the cases where:

\begin{enumerate}
	\item The output type of the function that is passed as an argument is more general (superclass) than the output type of the target function
	\item The argument type of the function that is passed as an argument is more specific (subclass) than the argument type of the target function
\end{enumerate}

This is a case where covariance and contravariance of the type arguments of the functions that our \lstinline|or()| method accepts should be taken into account. Let's consider the following type of the first function:

\lstset{language=Scala}
\begin{lstlisting}
(A) => B
\end{lstlisting}

This function should be combinable using \lstinline|or()| method with any 

\subsection{Generic methods}

%TODO: Solution suggestion - OOP approach, genericity in classes

%TODO: How to solve this issue?

Functions have one major limitation in Scala - they can't have any generic arguments. The signature of a function always has all the argument and return value types specified at compile-time. If we perform the eta-expansion on a generic method, we have to specify the type argument, otherwise, the type arguments will be fixed as Nothing and the function won't be callable:

\lstset{language=Scala}
\begin{lstlisting}
def makeTuple[A, B](a: A, b: B): (A, B) = (a, b)
val fun1 = makeTuple _
// fun1: (Nothing, Nothing) => (Nothing, Nothing)
val fun2 = makeTuple[Int, String] _
// fun2: (Int, String) => (Int, String)
\end{lstlisting}

As the \lstinline|or()| method performs combination of functions into another function, we automatically lose all the generic types involved in the original methods we are expanding and then combining. We can, however, take advantage of the fact that the type arguments being explicitly set in the process of eta-expansion can be generic types of an enclosing structure (generic class or generic method). With this approach, there are two possible patterns of achieving genericity in combined functions:

\begin{enumerate}
	\item Defining a function field inside a generic class
	\lstset{language=Scala}
	\begin{lstlisting}
	def defaultCount[A](list: List[A], item: A) = list.count(_ == item)
	def customCount[A](list: List[A], item: A) = list.filter(_ == item).map((i) => 1).sum
	
	class ListTools[A] {
	val count = defaultCount[A] _ or customCount[A]
	}
	\end{lstlisting}
	\item Defining a generic method by creating the combination and then calling it immediately
	\lstset{language=Scala}
	\begin{lstlisting}
	def count[A](list: List[A], item: A) = (defaultCount _ or customCount)(list, item)
	\end{lstlisting}
\end{enumerate}


Generics solution:
\lstset{language=Scala}
\begin{lstlisting}
def defaultMap[A, B](list: List[A], fun: (A) => B): List[B] = 
list.map(fun)
def iterativeMap[A, B](list: List[A], fun: (A) => B): List[B] =  {
val result = new mutable.MutableList[B]()
for (x <- list) {
result += fun(x)
}
result.toList
}

def map[A, B](list: List[A], fun: (A) => B): List[B] = 
(defaultMap[A, B] _ or iterativeMap[A, B])(list, fun)
\end{lstlisting}

\subsection{Implicit arguments}

Implicit arguments are arguments that will be filled in automatically at invocation by an implicit method with matching signature that is available within the scope. This can be used for different reasons, but the most common is a combination with a type argument that has to have a conversion of some kind. In the Scala terminology, the type has to be \textit{viewable} as some other type. There used to be a specialized syntax called \textit{View bounds} in earlier versions of Scala to support this usage, but it has been deprecated.

%TODO: http://docs.scala-lang.org/tutorials/tour/implicit-parameters.html

\lstset{language=Scala}
\begin{lstlisting}
def bubbleSort[A <% Ordered[A]](list: List[A]): List[A] = ???
def bubbleSort[A](list: List[A])(implicit ord: A => Ordered[A]): List[A] = ???
\end{lstlisting}

The most common usage of implicit arguments is simulating \textit{type classes} from other functional languages like Haskell. We create a generic trait with required methods (representing the type class and its functions) and whenever we want a new type to be member of the type class, we create implementation of the trait and an implicit method that provides the implementation. From now on, the newly added member can be used in any method requiring the implicit conversion just by importing our conversion function into the scope where the method is called.

%TODO: Example of custom "type class" or member?

As with a lot of other features, implicit arguments are supported only by methods, not by functions. Therefore, all of them have to be resolved and fixed when performing eta-expansion, along with all the type arguments. If we set the type arguments to a concrete type, we can provide the implementation in the same way as at the time of invocation, without any problem:

\lstset{language=Scala}
\begin{lstlisting}
def radixSort(list: List[Int]): List[Int] = ???
def bubbleSort[A <% Ordered[A]](list: List[A]): List[A] = ???

val sort = radixSort _ or bubbleSort[Int]
\end{lstlisting}

%TODO: Reference
There is, however, one more option mentioned in [REFERENCE] - the type argument can be set to a type argument of an enclosing scope, making the implicit argument generic again. In this case, the scope in which the type argument is declared has to provide the implicit function implementation. As a result, the implicit argument has to be repeated in the method or class constructor signature.

\lstset{language=Scala}
\begin{lstlisting}
def implicitFun1[A](list: List[A])(implicit ord: A => Ordered[A]): List[A] = ???
def implicitFun2[A](list: List[A])(implicit ord: A => Ordered[A]): List[A] = ???

// Following line won't compile:
// def implicitFun[A](list: List[A]): List[A] = (implicitFun1[A] _ or implicitFun2[A])(list)
def implicitFun[A](list: List[A])(implicit ord: A => Ordered[A]): List[A] = (implicitFun1[A] _ or implicitFun2[A])(list)d
\end{lstlisting}

\lstset{language=Scala}
\begin{lstlisting}
def bubbleSort[A](list: List[A])(implicit ord: A => Ordered[A]): List[A] = ???
\end{lstlisting}

\lstset{language=Scala}
\begin{lstlisting}
def implicitFun1[A <% Ordered[A]](list: List[A]): List[A] = ???
def implicitFun2[A <% Ordered[A]](list: List[A]): List[A] = ???

def fun = implicitFun1 _ or implicitFun2
\end{lstlisting}

Interesting thing is that the parser included in the IntelliJ IDEA IDE\footnote{Integrated Development Environment.} Scala plugin that performs background code inspection and immediately highlights compile-time errors doesn't recognize this problem. It was tested in version 2016.3.4 with Scala plugin version 2016.3.8.

%TODO: IntelliJ parser has no problems, compilation shows error

\lstset{language=Scala}
\begin{lstlisting}
def fun: (List[A]) => List[String] = implicitFun1 _ or implicitFun2
\end{lstlisting}

%TODO: Automatically generated type annotation is WRONG!!!

\subsection{Overloading}
Very similar issue is encountered when it comes to overloading - only methods can be overloaded.

%TODO: implementing traits using combined methods
\subsection{Implementing Traits}
Traits in the role of interfaces represent key element of object-oriented programming approach. Providing custom implementations of traits is essential and using adapted functions in this role would be very useful. The key problem is the separation of functions and methods in Scala mentioned in [REF].
%TODO: ADD REFERENCE HERE

The methods defined in traits can be implemented or overridden only by a different method, which will be invoked using virtual method calls. The function defined in the trait using a val keyword is basically a getter which returns the function value, invokable by itself. In this case, the getter has to be overridden, either by a custom getter, or by a field with automatically generated getter. An adapted function generated using the \lstinline|or()| method can be assigned to such a field and thus implementing the trait function getter. An example follows:

\lstset{language=Scala}
\begin{lstlisting}
trait TestTrait {
def testMethod(arg: List[Int]): List[String]
val testFunction: (List[Int]) => List[String]
}

class TestImpl extends TestTrait {
import functionadaptors.Implicits._

def impl1(arg: List[Int]): List[String] = 
arg.map("Num: " + _.toString)
def impl2(arg: List[Int]): List[String] = 
arg.map(i => s"Num: $i")

override val testFunction: (List[Int]) => List[String] = 
impl1 _ or impl2

// Can't implement testMethod using the result of or()
override def testMethod(arg: List[Int]): List[String] = ???
}
\end{lstlisting}

Unfortunately, a lot of the traits we need to provide implementations for are already existing and can use the method format. Manual workaround to implement the method is quite straightforward, but again, it generates unnecessary calls and duplicities in our code:

\lstset{language=Scala}
\begin{lstlisting}
...
private val adaptedFunction = impl1 _ or impl2
override def testMethod(arg: List[Int]): List[String] = 
adaptedFunction(arg)
...
\end{lstlisting}

%TODO: Add reference
Or, using the solution that was previously mentioned in [PREVIOUS SECTION], we can even omit the private field declaration and use an expression syntax:

\lstset{language=Scala}
\begin{lstlisting}
override def testMethod(arg: List[Int]): List[String] = 
(impl1 _ or impl2)(arg)
\end{lstlisting}


%TODO: summary of the API approach, definition of key terms and points in adaptive lifecycle - method definition, function / closure creation (eta expansion), function combination, 

\section{Identifier selection}

The functions that we are going to select from have to be identifiable in the entire application. More specifically, we need to be able to store the runtime measurements for every function throughout the entire execution process. The functions might be referenced and called from various points in the library user's code. Additionally, we know nothing about the runtime state of the application memory at the point of entry into our function. So the auxiliary data have to be stored statically, in a special memory section dedicated exclusively to the framework, and the access has to be thread safe.

%TODO: Thread safety

The historical runtime measurements (run history) of a function can be identified using various approaches.

\subsection{Arbitrary identifiers}
User of the library could assign custom identifiers to the functions. Main advantage of this approach is the possibility to include description or documentation of the specific implementation in the identifiers. For example:

\begin{lstlisting}
quickSort
heapSort
\end{lstlisting}

\begin{lstlisting}
joinQuery
selectInQuery
\end{lstlisting}

Having meaningful identifiers leads to readable logs or measurement history dumps and to easier debugging in general. It, however, imposes new requirements on the user of the library and makes the API more complex. In addition, the identifiers would in almost all the cases exactly match the name of the function implementation.

\subsection{Randomly generated identifiers}
If we abandoned the idea of meaningfulness, we could assign automatically generated identifiers without the user having to specify them. Possible candidates could be sequential numbers or GUIDs. The identifiers would have to be created and matched with the function definitions or combi

%TODO: Evaluation, advantages, disadvantages, generation

\section{Delayed measuring}

%TODO: Use cases - configuration generation, expression tree building, query building...

%TODO: Problems - matching decision with measurement
\chapter{Performance evaluation}

\section{Data obtained by measurement}

\section{Evaluation ???}

\section{Selection strategies within a bucket}

\subsection{Statistical tests}

\section{Cross-bucket selection}

\section{Experimental behavior with performance limits}
Using available time to "obtain" more data

%TODO - when do I need more data?
\chapter{ScalaAdaptive library implementation}
\label{chap:implementation}

\section{Goals of implementation}

The main concerns upon implementing the library were the following:

\begin{itemize}
	\item To separate the API from the selection logic
	\item To keep the library strongly typed and use static type checking even in the internals
	\item To make the library as extensible as possible
	\item To keep the library overhead as low as possible
	\item To use the Scala language features to make the code simpler and easier to maintain
\end{itemize}

\subsection{Development approach}

\begin{itemize}
	\item Prefer immutable structures
	\item Prefer composition before inheritance
	\item Use functional approach
	\item In composition, use only trait types
	\item Prefer Optional passing (railway oriented programming) for error handling
\end{itemize}

\subsubsection{Error handling}

The library is designed so that the number of exceptions handled in the code was minimized. The library itself doesn't raise any exceptions in case of errors, and catches most of the exceptions from the libraries within to replace them with a None return value.

This approach is known from the functional programming and takes advantage of monadic operations over the \textit{Option monad}\footnote{In Haskell and other languages known as Maybe monad.}. The return values can be mapped over using the \textit{bind} operator, which allows smooth function chaining and the error propagation through the chain.

\section{Architecture overview}

This section will provide brief overview of the entire framework architecture and related decisions. More detailed description at the level of individual classes can be found in the documentation.
%TODO: Documentation reference

\subsection{API architecture}

As briefly described in section \ref{subsec:api_implementation}, the public API of the framework is represented by a set of traits \inlinecode{MultiFunction0}, ..., \inlinecode{MultiFunction5}, which expose all the user accessible methods. Because only some of the methods depend on the number of input arguments, the part of the API that can be generalized was extracted into a common trait \inlinecode{MultiFunctionCommon}, which is parametrized by the function type that extends it. 

The \inlinecode{MultiFunctionN} traits are implemented by a set of classes called \inlinecode{FunctionAdaptorN}. This is the default implementation directly inaccessible to the user. This separation allows us to modify or extend the internal API of the \inlinecode{FunctionAdaptorN} classes without changing the public API. The methods that can be implemented independently on the number of input arguments are implemented in an abstract class \inlinecode{FunctionAdaptorBase} that the other classes inherit from.

It would be quite impractical to manipulate with N different classes that represent the combined functions inside the internal parts of the framework. Therefore, the actual class holding the functions, \inlinecode{CombinedFunction}, is wrapped inside each one of the \inlinecode{FunctionAdaptorN} and is parametrized by the tuple type consisting of all the N arguments of the original function, and by the return type. All the functions are converted into functions of type \inlinecode{(TArgType) => TRetType} in the following way:

\lstset{style=Scala}
\begin{lstlisting}
val tupleFunction = (tupleArg) => function(tupleArg._1, ..., tupleArg._n)
\end{lstlisting}
	
The whole inner part of our framework works only with these \inlinecode{CombinedFunction} types and therefore with functions of type \inlinecode{(TArgType) => TRetType}. Individual function adaptor classes perform the composition of the arguments into the tuple version before propagating the call further whenever a method depending on the argument number is invoked.

The figure \ref{fig:inheritance_function_adaptors} shows this inheritance scheme.

\begin{figure}[h!]
	\captionsetup{justification=centering,margin=0.5cm}
	\centerline{\mbox{\includegraphics[width=140mm]{./img/inheritance_function_adaptors.png}}}
	\caption{Diagram showing the inheritance chain of MultifFunctionN and related classes.}
	\label{fig:inheritance_function_adaptors}
\end{figure}

\subsection{Internal architecture}
\label{subsec:internal_architecture}

The simplified internal architecture can be seen in the figure \ref{fig:internal_architecture}. It shows the core chain that is executed when a combined function, represented as a \inlinecode{MultiFunctionN} to the user, but internally stored as a \inlinecode{CombinedFunction} instance, is called.

As we can see, there are two principal logic units participating in the process:

\begin{enumerate}
	\item \textbf{\inlinecode{CombinedFunctionInvoker}} \\
	Supposed to make a quick decision based on the policy of the combined function. It should either directly invoke one of the functions, or delegate the call to the \inlinecode{AdaptiveSelector}, and update the statistics of the combined function afterwards. It works only with the data directly stored in the \inlinecode{CombinedFunction} instance, it does not have any state or shared storage.
	\item \textbf{\inlinecode{AdaptiveSelector}} \\
	The key component of the framework, it receives a set of functions and an input, and it should decide which one of them to run and to execute it, evaluate the function run and return the result. The default implementation works with an abstract history storage (\inlinecode{HistoryStorage}) a pair of selection strategies (\inlinecode{SelectionStrategy}) and an evaluator (\inlinecode{EvaluationProvider}).
\end{enumerate}

\begin{figure}[h!]
	\captionsetup{justification=centering,margin=0.5cm}
	\centerline{\mbox{\includegraphics[width=140mm]{./img/internal_architecture.png}}}
	\caption{Diagram showing the internal architecture of the ScalaAdaptive framework.}
	\label{fig:internal_architecture}
\end{figure}

\section{Implementation options}

\subsection{HistoryStorage location}
\label{subsec:storing}

\inlinecode{HistoryStorage} is supposed to gather and store history data for all functions that are invoked using a combined function. The question is, where to save the data and how accessible to make it.

\subsubsection{Local}

First option is to make the storage part of the \inlinecode{CombinedFunction} itself. In this case, the history data would, just like the statistics, be unique for every instance of the class, i.e. for every \inlinecode{MultiFunctionN} created by the user of the library. This has two consequences:

\begin{itemize}
	\item A function will have separate histories for every use in a combined function
	\item Every new instance of a combined function will have an entirely new history for all the functions
\end{itemize}

The second consequence might lead to some unexpected behavior - for example, if we create a combined function inside of a method call as described in section \ref{subsubsec:method_from_combined_func} and call it just once, it will have a clear history every time and will be useless:

\lstset{style=Scala}
\begin{lstlisting}
def processData(data: List[Int]): Int = 
  (impl1 _ or impl2 withStorage Storage.Local)(data)
\end{lstlisting}

Some use cases for this configuration exist, e.g. if we have a class that holds some immutable data and we keep executing some computations on the data, then a combined function defined as a field on this class with local storage will adapt specifically to the data of the class and does not need predictive strategies.

\subsubsection{Global}

Another option is to store all the measured data globally, in a static area of the memory accessible from all contexts. This will allow to collect run data for functions faster and share them across all the combined functions, which usually leads to more informed and more precise decisions. This is usually the preferred approach for some business logic or utility methods in long-running services. A unique identifier has to be used to match the function history record with the actual function in the \inlinecode{CombinedFunction} class. The selection of the identifier will be discussed later.

\subsubsection{Persistent}

The biggest problem of the discussed storage location is that the run history data are available only during a single run of the application. Whenever the application restarts, it will have to collect the data again. This might not be a problem for some long running services or daemons, but it is not very convenient for some tools or client applications with shorter lifecycle.

A solution to the problem is storing the data in a persistent storage, most likely on the HDD. We can do it in three different ways:

\begin{enumerate}
	\item Immediately persist each run history record
	\item Persist all of the records when the application terminates
	\item Buffer the history records and persist them in batches
\end{enumerate}

The first option seems like the best one, but it would lead to a series of I/O operations upon every invocation of a combined function. The added overhead of this solution would be dramatic. The second option, on the other hand, has no runtime overhead at all, but it requires a possibility to detect an application termination from our framework. JVM does not guarantee finalizers being called and relying on notifications from the user would complicate the API and generally be problematic. There is a mechanism in JVM called \textit{shutdown hooks} that allow the user to perform a custom action on shutdown, but it will not be called when the application crashes. Additionally, it might be unexpected from a library to perform actions on shutdown and the delay that caused by persisting a lot of run data on a slow HDD (or even a network drive) could lead to the application being terminated immediately in some automatized environments.
Therefore, the third option was selected, as it leads to the runs being regularly saved and the I/O overhead is lower.

This solution relies on the unique function identifier as well. There are additional problems connected to persisting the run history, namely:
\begin{itemize}
	\item Running multiple instances of the application at once (collisions on the persisted data file)
	\item Changes in the application code (outdated run data, changed identifiers, etc.)
	\item Having to deal with larger amounts of history data
\end{itemize}

\subsection{History storage function identifiers}
\label{subsec:function_identifiers}

As described in section \ref{subsec:storing}, we need some sort of identifier of the function to be able to store the run history data in a global or persistent storage. It does not make sense to use the references to the function objects, as new closures are created every time a lambda expression (or an eta-expanded method) is assigned.

%TODO: Reference an example?

This leaves us with three basic identifier options.

\subsubsection{Type name}

Functions in Scala are instances implementing the \inlinecode{FunctionN} trait. The default implementations are anonymous closure classes that are compiled from lambda expressions. Two different functions originated in two different lambda expressions and thus have two different type names, the ones that were generated and assigned by the compiler. The fully qualified type name can be used as the identifier of a function

Using the type names as unique identifiers is safe and straightforward. A small disadvantage is that they aren't very readable, the compiler uses the name of the type that contained the lambda expression followed by sequential number. 

There is also a subtle danger connected - in case of persisting the run history, the closure classes might get renamed automatically upon recompiling. The compiler usually assigns the closure names sequentially, so this could happen by just inserting another lambda expression into the enclosing class code before the current one. Run history data might even get mixed up as the newly added lambda expression could have the former name of the original closure (by taking its position in the sequence).

If the \inlinecode{FunctionN} trait has a different than default implementation, the type name identification might fail - the trait can be implemented by a single class wrapping other values. In this case, all functions implemented by that class would share the run history.

Last but not least, every time a method gets eta-expanded, a new lambda expression with new type name is generated, so in the following case, the \inlinecode{method1} would not have the same identifier in the two combined functions:

\lstset{style=Scala}
\begin{lstlisting}
val fun1 = method1 _ or method2
val fun2 = method1 _ or method3
\end{lstlisting}

\subsubsection{Method name}
\label{subsec:methodnameident}

The most common usage pattern is the one where the functions used with the \inlinecode{or} method are \textit{eta-expanded} methods (see \ref{subsubsec:apimethods}). It would be handy to use the method name as an identifier in such a case, which would lead to a better readability and allow the same method used in different combined functions to have just one run history.

The problem is that at runtime, the implicit type conversion method or the \inlinecode{or} method will always get the already \textit{eta-expanded} function object with a compiled \inlinecode{apply} method. The names have to be extracted at compile time, using the def macros (see \ref{sec:defmacros}). At the moment of implicit conversion from \inlinecode{FunctionN} to \inlinecode{FunctionAdaptorN}, the AST\footnote{Abstract syntax tree.} of the \inlinecode{FunctionN} expression can be examined - if it's a lambda expression containing one method call, the method name can be extracted.

\subsubsection{Custom identifier}

In order to handle the specific cases where type name and method name identifiers do not distinguish correctly between different functions, there is also a possibility of choosing a custom, arbitrary identifier. This has to be triggered specifically by the user in the API and should be used in the cases where:

\begin{enumerate}
	\item User knows that the automatically assigned identifiers will not be sufficient
	\item User wants to replace default type identifier with custom identifier because of readability
\end{enumerate}

\section{Extracting method name from eta-expansion AST}

We would like to extract the method name in the process of conversion from \inlinecode{FunctionN} to the \inlinecode{MultiFunctionN}. To achieve that, we are going to have to modify the conversion process in the following way:

\begin{enumerate}
	\item Replace the implicit conversion method from \inlinecode{FunctionN} to \inlinecode{FunctionAdaptorN} by a def macro (see \ref{sec:defmacros}) and extract the conversion logic into a different, non-implicit method (in our case, the Conversions singleton object methods \inlinecode{toAdaptor()})
	\item Inside the macro, analyze the AST, detect eta-expansion method call
	\item If there is a method call:
	\begin{enumerate}
		\item Generate the identifier expression as a \inlinecode{getTypeName} call on the target of the method call, followed by the method name
		\item Generate the conversion code with explicitly specified reference expression
	\end{enumerate}
	\item Otherwise generate the conversion code with implicit reference	
\end{enumerate}

The conversion is done using the \inlinecode{toAdaptor()} method with two overloads:
\begin{itemize}
	\item Accepting only the function - implicit reference is used (type name of the closure)
	\item Accepting the function and a custom reference - the reference provided is used
\end{itemize}

\subsection{Eta-expansion AST format}

First step in the macro implementation has to be parsing the input AST and detecting patterns that are generated from eta-expansions by the compiler. Using the \inlinecode{printAst()} macro mentioned in \ref{subsec:buildingast}, the following facts were discovered:

\begin{itemize}
	\item The eta-expansion is already replaced by the equivalent code in AST, so it can't be detected directly.
	\item The result of eta-expansion is a lambda expression (function literal).
\lstset{style=Dump}
\begin{lstlisting}
Function(...)
\end{lstlisting}
	\item The lambda expression is always wrapped in a block, being its return value.
	
\lstset{style=Dump}
\begin{lstlisting}
Block(
  List(...), 
  Function(...))
\end{lstlisting}	
	
	\item If the target of the invocation is either a constant or \textit{this}, it is captured in the lambda expression closure (i.e., the constant or \textit{this} is referenced directly from the function body).
	
	\item If the target of the invocation is a variable or a result of a more complicated expression, it is extracted to the enclosing block, its result is stored in a variable local to the block and then captured in the lambda expression closure. The reason is probably to avoid multiple evaluation of expressions with possible side-efects upon every invocation of the resulting function, and to avoid the target being changed throughout the lifetime of the function. The following example shows the target being a result of an expression \inlinecode{this.getInstance()} on a \inlinecode{Class} class.
	
\lstset{style=Dump}
\begin{lstlisting}
Block(
  List(
    ValDef(
      Modifiers(SYNTHETIC), 
      TermName("eta$0$1"), 
      TypeTree(), 
      Apply(
        Select(
          This(
            TypeName("Class")), 
          TermName("getInstance")), 
        List()))), 
  Function(...))
\end{lstlisting}

	\item The function node contains argument definition and the expression itself, which is a single function application.

\lstset{style=Dump}
\begin{lstlisting}
Function(
  List(
    ValDef(
      Modifiers(PARAM | SYNTHETIC), 
      TermName("arg"), 
      TypeTree(), 
      EmptyTree)), 
  Apply(...))
\end{lstlisting}
\end{itemize}

This has a few consequences for our case. We need to generate our conversion and method retrieval code into the block return value, because we need to be able to access the invocation targets that can be defined in the block itself. The target and the method name will always be in the single Apply node representing the function body.

If the method doesn't accept any type arguments, the tree is quite simple:

\lstset{style=Dump}
\begin{lstlisting}
Apply(
  Select(
    ...invocation target expression..., 
    TermName("methodName")), 
  List(...function arguments...))
\end{lstlisting}

Where the \textit{invocation target expression} can have multiple forms based on the original expression, and can depend on the enclosing block variables. It isn't, however, important for us, as we can work with the expression as whole. The function arguments are not needed either.

If the method is generic, the type arguments need to be applied in order to convert it to a function (which can't be generic). In this case, the tree gets a little more complicated:

\lstset{style=Dump}
\begin{lstlisting}
Apply(
  TypeApply(
    Select(
      ...invocation target expression..., 
      TermName("genericMethod")), 
    List(...type arguments...))), 
  List(...function arguments...))
\end{lstlisting}

The method call is wrapped in a TypeApply node before being invoked using the Apply node. The TypeApply node can in our case be ignored.

And the most complicated case we can encounter is when the method has some implicit arguments as well:

\lstset{style=Dump}
\begin{lstlisting}
Apply(
  Apply(
    TypeApply(
      Select(
        ...invocation target expression...,
        TermName("genericMethodImplicit")), 
      List(...type arguments...)), 
    List(...function arguments...)), 
  List(...implicit arguments...))
\end{lstlisting}

One more Apply node is added to the topmost level - the implicit arguments are applied after applying the actual function arguments. Their definition contains another nested block and lambda expression, but again, it is not necessary for our case. We just need to extract the invocation target and the method name.

\subsection{Generating the conversion}
\label{subsec:generate_conversion}

Supposing we have the invocation target expression and the method name, we need to create the identifier string expression that will be used in the manual \inlinecode{toAdapter} invocation.

In order to extract the fully qualified name of the method call target, we need to generate the following expression:

\lstset{style=Scala}
\begin{lstlisting}
invocationTarget.getClass.getTypeName + ".methodName"
\end{lstlisting}

The AST representing this expression has to be wrapped in the \inlinecode{MethodNameReference} construction application (it is a case class with automatically generated \inlinecode{apply()} method) to get the resulting reference.

In all the cases, the \inlinecode{toAdapter} method has to be generated to replace the macro function. The function literal from the original AST has to be passed in as the first argument, and optionally, the second argument can be provided if the extraction of the method name was successful.

The original tree in a simplified form can be seen in figure \ref{fig:original_eta_ast}.

\begin{figure}[h!]
	\captionsetup{justification=centering,margin=0.5cm}
	\centerline{
		\begin{forest}
			[Block
			[Statements
			[ValDef]
			]
			[Function]
			]
		\end{forest}
	}
	\caption{The original simplified eta-expansion AST.}
	\label{fig:original_eta_ast}
\end{figure}

We need to transform the block, keeping the definition in the statement part, but wrapping the \inlinecode{Function} into the \inlinecode{toAdapter} call, as shown in figure \ref{fig:transformed_eta_ast}.
	
\begin{figure}[h!]
	\captionsetup{justification=centering,margin=0.5cm}
	\centerline{
		\begin{forest}
			[Block
			[Statements
			[ValDef]
			]
			[Apply
			[toAdaptor]
			[ArgumentList
			[Function]
			[ExtractedReferenceExpression]
			]
			]
			]
		\end{forest}
	}
	\caption{The simplfied eta-expansion AST after manually adding toAdaptor call.}
	\label{fig:transformed_eta_ast}
\end{figure}

\subsection{Extracting method overloads}

The approach that was described so far has one small issue - it doesn't recognize function overloads, so all the overloads share the same identifier.

 It wouldn't be difficult to extract the actual number of arguments that the method is being invoked with and include it in the reference. When attempting to extract the argument types of the lambda expression, we encounter a major problem - the function literal is generated without argument type specification, the TypeTree is empty:

\lstset{style=Dump}
\begin{lstlisting}
ValDef(
  Modifiers(PARAM | SYNTHETIC), 
  TermName("i"), 
  TypeTree(), 
  EmptyTree)
\end{lstlisting}

This can be done and is quite similar to the following piece of code:

\lstset{style=Scala}
\begin{lstlisting}
val function: (Int) => Int = { i => math.abs(i) + 1 }
\end{lstlisting}

In this case, the lambda expression doesn't have the type of its arguments specified either, because the compiler will infer it from the context in which the expression is used, in this case, from the type specifier of the variable.

The compiler is able to infer the data type from the usage of the argument as well:
\lstset{style=Scala}
\begin{lstlisting}
def method(i: Int): Int = ???
val function = { i => method(i) }
\end{lstlisting}

The previous piece of code is correctly compiled by the Scala compiler, although some IDEs (namely IntelliJ IDEA 2016.3.4) aren't able to infer the type and mark the code as incorrect.

Note that the eta-expansion is guided by the same rules, so whenever expanding a method without overloads, the compiler infers the types by itself:
\lstset{style=Scala}
\begin{lstlisting}
def method(i: Int): Int = ???
val function = method _
\end{lstlisting}

Upon expansion of a method with overloads, the resulting function type has to be provided and the inference flow goes in the other direction:
\lstset{style=Scala}
\begin{lstlisting}
def method(i: Int): Int = ???
def method(s: String): Int = ???
val function: (String) => Int = method _
\end{lstlisting}

As a consequence, at the time of syntax analysis, the types aren't inferred yet and there is no way for us to find out which method overload is being called.

The method overloads have to share the same identifier, but in most typical cases, this isn't a problem. Otherwise, it can be solved by using custom identifiers.

\subsection{The conversion demonstration}

As a practical demonstration of the process described in this section, we are going to show the compile-time conversions that are done on a method identifier that is being \textit{eta-expanded} and converted into a \inlinecode{MultiFunctionN}. Let's consider the following code:

\lstset{style=Scala}
\begin{lstlisting}
val combined1 = target.method _ or function2
\end{lstlisting}

We will follow the conversions that the compiler and our macro perform on the \inlinecode{target.method \_} expression\footnote{The actual changes are performed on the AST level, we will show them in corresponding Scala code for simplicity.}:

\begin{enumerate}
	\item The explicit \textit{eta-expansion} is done:
	\lstset{style=Scala}
	\begin{lstlisting}
{ () => target.method() }
	\end{lstlisting}
	\item The implicit conversion is applied:
		\lstset{style=Scala}
	\begin{lstlisting}
Implicits.toMultiFunction0({ () => target.method() })
		\end{lstlisting}
		\item The macro \inlinecode{toMultiFunction0} is executed and expanded:
		\lstset{style=Scala}
		\begin{lstlisting}
Conversions.toAdaptor({ () => target.method() }, MethodNameReference(this.getClass.getName + ".method"))
		\end{lstlisting}
\end{enumerate}

The final state shows the code that will be present in our compiled program and will be executed.

\subsection{Macros in the combining method}

In the section \ref{subsubsec:apimethods}, we explained why the \inlinecode{or} method has to accept an argument of \inlinecode{FunctionN} type instead of \inlinecode{MultiFunctionN}. This does not seem to cause any trouble as we can manually convert \inlinecode{FunctionN} into the \inlinecode{FunctionAdaptorN} inside the method body. With the macro expansion, however, we run into problems. Image we do it the following way:
\lstset{style=Scala}
\begin{lstlisting}
def or(fun: (T1) => R) = orMultiFunction(Implicits.toMultiFunction1(fun))
\end{lstlisting}

The  \inlinecode{toMultiFunction1} method is a macro - it gets executed at compile time, and it receives the argument AST as its input. In this case, an AST that consists of the \inlinecode{fun} identifier. We have no way to find out, where this function will be called from and with which arguments, so in order to extract the method name, we need to convert the entire \inlinecode{or} function to a macro which will generate the conversion code just like described in section \ref{subsec:generate_conversion}, and then wrap it in the internal \inlinecode{orMultiFunction} code.

This does not require any extra AST manipulations, only one method call generations. It does, however, mean a complication - the macros cannot be called virtually. Therefore, the macro definition has to be part of the \inlinecode{MultiFunctionN} API trait, breaking somehow the encapsulation of the implementation. Fortunately, it does not perform any actions by itself, only replaces the \inlinecode{or} call with a conversion and an \inlinecode{orMultiFunction} call, which is, again, a virtual method defined on the trait without implementation and therefore encapsulated.


\section{Module implementation}
\label{sec:module_impl}

All the modules mentioned in \ref{subsec:internal_architecture} are abstract traits with replaceable implementations, that are put together using composition. The API classes have to be able to access the composed and active implementations at the moment of invocation. For this reason, a singleton Scala object \inlinecode{AdaptiveInternal} was introduced and provides a holder for the following composed modules:

\begin{itemize}
	\item AdaptiveSelector with global storage
	\item AdaptiveSelector with persistent storage
	\item CombinedFunctionInvoker
\end{itemize}

We will briefly introduce the implementations of the modules that are included in the framework.

\subsubsection{CombinedFunction}

The \inlinecode{CombinedFunction} is the main data class representing a function with multiple implementations in ScalaAdaptive. It is basically a data holder class, the important functionality was separated in order to be possible to replace. It gets created and modified when a user of the library manipulates with the \inlinecode{MultiFunctionN} API type. Part of its attributes are immutable, representing the configuration and default setup of the function, and part of its attributes are mutable, holding its state in the invocation process.

It contains the following immutable data:

\begin{itemize}
	\item Set of functions that it is combined from (wrapped into the tuple form)
	\begin{itemize}
		\item	Each function holds up to two references - one based on the closure type name, the other being either a method name or a custom name, depending on the creation process (see \ref{subsec:function_identifiers})
	\end{itemize}
	\item \textit{Descriptor function} (wrapped into the tuple form), see \ref{sec:predictive_strategies}
	\item \textit{Group selector} function (wrapped into the tuple form), see \ref{subsec:group_selection}
	\item A \inlinecode{FunctionConfiguration} a setup independent on the argument and return value types, consisting of the following:
	\begin{itemize}
		\item Selection strategy - Predictive or NonPredictive
		\item Storage - Local, Global or Persistent
		\item Duration - optional maximal time period for the history records
		\item ClosureReferences - if the closure type references should be preferred over the method name or the custom name
		\item Initial policy
	\end{itemize}
\end{itemize}

And the following state representing data:

\begin{itemize}
	\item Function statistics (separately for each group)
	\item Current policy (separately for each group)
	\item Analytics data - full history of all the selections and their results
\end{itemize}

In case of local storage setup, the instance also contains its own \inlinecode{AdaptiveSelector} that holds the local \inlinecode{HistoryStorage}.

\subsubsection{CombinedFunctionInvoker}

It has access to the state of the \inlinecode{CombinedFunction} and its basic implementation does the following:

\begin{itemize}
	\item Evaluate the current policy on the statistics of the combined function
	\item According to the result either invoke directly the function which is directly accessible through the statistics (the last one and the most selected one), or pass the decision onto the \inlinecode{AdaptiveSelector}
	\item Update the function statistics
\end{itemize}

The \inlinecode{AdaptiveSelector} is accessed either through the \inlinecode{CombinedFunction} itself in case of local storage setup, or using the singleton object \inlinecode{AdaptiveInternal} in case of global or persistent setup.

\subsubsection{AdaptiveSelector}

A module which is supposed to select one of given options to run, and to evaluate the run. It is parametrized with \inlinecode{TMeasurement} type, which represents the data measured from the function run. By default, a \inlinecode{Long} type for run time measured is used.

The implementation \inlinecode{HistoryBasedAdaptiveSelector} uses a \inlinecode{HistoryStorage} to store and retrieve the run data of individual functions. In addition, it holds two \inlinecode{SelectionStrategy} instances, one for the predictive selection, the other for the non-predictive selection. One of these two instances is used to select the function to run, which is then executed using an \inlinecode{EvaluationProvider}. The measurement retrieved is added to the history and the result is returned, along with a performance benchmark including execution and overhead times (independent on the actual measurement, always based on wall-clock times) for the function statistics update. Figure \ref{fig:history_based_selector} shows the entire process.

\begin{figure}[h!]
	\captionsetup{justification=centering,margin=0.5cm}
	\centerline{\mbox{\includegraphics[width=100mm]{./img/history_based_selector.png}}}
	\caption{Diagram showing the HistoryBasedAdaptiveSelector execution path.}
	\label{fig:history_based_selector}
\end{figure}

\subsubsection{HistoryStorage}

A type with a map-like interface - it is supposed to hold \inlinecode{RunHistory} instances for each combination of function and group. A key composed of the function reference and the group identifier is used to access the histories.

There are two implementations in the framework:

\begin{itemize}
	\item \inlinecode{MapHistoryStorage} - stores the histories in memory
	\item \inlinecode{PersistentHistoryStorage} - a wrapping storage that delegates the calls to an internal \inlinecode{HistoryStorage}, and, in addition, it serializes every new run to a \inlinecode{HistorySerializer} and if asked for a history that is not present in the underlying storage, it tries to deserialize it using the same class
\end{itemize}

The \inlinecode{HistorySerializer} has two basic implementations, one for direct serialization, and one for buffered serialization. The data are stored into a file, one per function, and the format, root directory and file name pattern are fully customizable. There is no list of these history containing files anywhere - when trying to deserialize a function, a file with corresponding name is checked for existence. This means that the individual function histories are deserialized one by one, at the moment of the first invocation of a combined function that contains them, which might cause a delay in the first execution.

\subsubsection{RunHistory}

\inlinecode{RunHistory} represents the sequence of measurement results and other details about a function run. The interface is designed to support immutable solution - the appending methods always return an instance of the same type, which might or might not be the same object depending on the implementation. The interface provides some basic operations on the history, allows iteration over the data and appending new data (always at the end). Apart from that, it also has some methods to support precomputed or cached data for some of the selection strategies, namely statistical data and grouped averages for each input descriptor. These supporting methods can always be computed using the data available, and if the implementation of \inlinecode{RunHistory} does not support the caching or precomputation, there are \textit{mixin} traits with default implementations available.

The provided implementations of \inlinecode{RunHistory} can be chained using the \textit{Decorator} pattern to modify the behavior. The basic implementations are the following:
\begin{itemize}
	\item \inlinecode{FullRunHistory} - stores all runs in an \inlinecode{ArrayBuffer}, is mutable
	\item \inlinecode{ImmutableFullRunHistory} - stores all runs in an immutable \inlinecode{List}, is a little slower in general
\end{itemize}

These instances can be wrapped in the following decorator classes:
\begin{itemize}
	\item \inlinecode{LimitedRunHistory} - limits the maximum number of stored items, whenever it reaches maximum, it throws the older half of all the records away
	\item \inlinecode{CachedStatisticsRunHistory} - stores the statistical data about the run items, speeds up some selection strategies
	\item \inlinecode{CachedGroupedRunHistory} - stores average evaluation data for each \textit{run selector}, speeds up some selection strategies
\end{itemize}

\subsubsection{SelectionStrategy}

Very simple trait that represents the selection strategies described in detail in sections \ref{sec:predictive_strategies} and \ref{sec:non_predictive_strategies}. The implementations should be stateless and should work only with the \inlinecode{RunHistory} records that it receives (in order not to mix together decisions about two separate history storage locations). The strategy implementations often have a customizable fallback strategy that is used in cases where the current strategy is not able to make a decision.

The strategies are parametrized by the \inlinecode{TMeasurement} type of the function run evaluation. All of the implemented strategies work with function run time represented by \inlinecode{Long}.

Implemented strategies are:
\begin{itemize}
	\item \inlinecode{LowRunAwareSelectionStrategy} - if any of the functions has less than a specified number of history records, uses one strategy, otherwise, uses a different strategy
	\item \inlinecode{LeastDataSelectionStrategy} - uses the function with least historical runs
	\item \inlinecode{TTestSelectionStrategy} - implements the t-test strategy described in section \ref{subsec:t_test_multiple}, uses the \inlinecode{org.apache.commons.math3} library
	\item \inlinecode{LimitedRegressionSelectionStrategy} - implements the window-bound linear regression strategy (the window being optional) as described in section \ref{subsec:window_bound_regression}, uses the \inlinecode{org.apache.commons.math3} library
	\item \inlinecode{LoessInterpolationSelectionStrategy} - implements the local regression strategy from section \ref{subsec:local_regression}, uses the \inlinecode{org.apache.commons.math3} library
\end{itemize}

The common composition pattern is to use the \inlinecode{LowRunAwareSelectionStrategy} as the topmost one, the \inlinecode{LeastDataSelectionStrategy} as the fallback strategy, and to have one of the actual predictive or non-predictive strategies in the middle.

\subsubsection{EvaluationProvider}

With a goal of potentially supporting larger variety of possible function evaluation criteria, the selected function is always executed using an implementation of \inlinecode{EvaluationProvider}, which is supposed to run the function and to evaluate it, and return both the result of the function and the evaluation data in the form of \inlinecode{TMeasurement}. The default implementation simply measures the wall-clock time of the function run.

\section{Configuration and customization}

The basic behavior of the combined functions can be changed in two ways:
\begin{itemize}
	\item Each combined function can have the most basic features configured
	\item The whole framework can be re-configured and customized
\end{itemize}

\subsection{Composed function configuration}

There is a fluent and simple way to configure the most basic behavior as part of the function composition API (see section \ref{subsec:api_function_configuration}). It is based mostly on selecting which one of multiple implementation options should be used with given function. In the \inlinecode{AdaptiveInternal} singleton, we hold two implementations of \inlinecode{AdaptiveSelector}, one with persistent history storage, the other without. Each of these two implementations then includes two instances of  \inlinecode{SelectionStrategy}, one for predictive and one for non-predictive selection.

The function configuration methods and possible values are the following:

\begin{itemize}
	\item \inlinecode{storeUsing} - represents the history storage options as described in section \ref{subsec:storing}
	\begin{itemize}
		\item \inlinecode{Global} - uses the shared \inlinecode{AdaptiveSelector} without persistent storage
		\item \inlinecode{Persistent} - uses the shared \inlinecode{AdaptiveSelector} with persistent storage
		\item \inlinecode{Local} - creates a new instance of \inlinecode{AdaptiveSelector} locally in the function object
	\end{itemize}
\item \inlinecode{selectUsing}
\begin{itemize}
	\item \inlinecode{Predictive} - uses the predictive strategy from the \inlinecode{AdaptiveSelector}
	\item \inlinecode{NonPredictive} - uses the non-predictive strategy from the \inlinecode{AdaptiveSelector}
\end{itemize}
\item \inlinecode{limitedTo} - allows to specify the maximum age of the history records to be used in the selection process
\item \inlinecode{asClosures} - represents the history storage identifier choice as described in section \ref{subsec:function_identifiers}
\begin{itemize}
	\item \textit{true} - always uses closure type names as the function references
	\item \textit{false} - uses either method names or custom names as the function references if available
\end{itemize}
\item \inlinecode{withPolicy} - allows to specify the starting policy for the function
\end{itemize}

The user of the library is expected to specify this configuration with most of the combined functions. Especially the choice between the predictive and non-predictive strategy is very important.

\subsection{Framework configuration}
\label{subsec:framework_config}

In section \ref{sec:module_impl}, we mentioned a variety of building blocks that the main parts of the framework are composed of, and the possibility to use multiple implementation to change the behavior of the entire system. This is the more advanced part of the configuration and it requires manipulation with the actual implementations.

As explained earlier, all the shared and commonly accessible functionality is held inside the singleton object \inlinecode{AdaptiveInternal}. This object can be initialized using an \inlinecode{initialize()} method on a publicly accessible singleton \inlinecode{Adaptive}. The static description of the composition of the implementations that is followed in this process is provided by a \inlinecode{Configuration} trait. This can be though of as a composition root of the whole system - it has to provide factory functions for all the building blocks used. In addition, it has an abstract type member \inlinecode{TMeasurement} for the evaluation data. 

%TODO: Add ref to http://docs.scala-lang.org/tutorials/tour/abstract-types.html

The implementation of the \inlinecode{Configuration} trait has to specify the \inlinecode{TMeasurement} type and provide the factory methods which are already specific for the type. Note that the \inlinecode{AdaptiveSelector} trait and its methods do not depend on the \inlinecode{TMeasurement}, only its implementation does, so the type of the evaluation data does not leak outside of the composition root. As a result, the actual type can be changed at runtime by simply replacing the \inlinecode{AdaptiveSelector} implementation.

The \inlinecode{Configuration} can be created by the user himself, using either a combination of provided implementation, or his own implementations, or, one of the predefined \inlinecode{Configuration} types can be used:

\begin{itemize}
	\item \inlinecode{BaseConfiguration} - provides basic configuration of the types that do not depend on the \inlinecode{TMeasurement}
	\item \inlinecode{BaseLongConfiguration} - sets the \inlinecode{TMeasurement} type to \inlinecode{Long} and add basic configuration of the types that depend on it
\end{itemize}

These traits can be extended and customized by overriding some of the functions.

\subsubsection{Configuration blocks}
 To simplify the configuration process when working with existing implementations and to hide the initialization details from the user, a concept called \textit{configuration blocks} was introduced. 
 
 \textit{Configuration blocks} are traits that provide implementation for just one (or a few) functions from the \inlinecode{Configuration} trait, setting up part of the framework in given way. The user of the framework can then create the configuration by creating an anonymous class extending the base configuration along with the \textit{configuration blocks} that specify the requirements. Thanks to the fluent and simple syntax of Scala, this concept is very expressive and easy to use
 
 The whole configuration can look the following way:

\lstset{style=Scala}
\begin{lstlisting}
val config = new BaseLongConfiguration
  with RunTimeMeasurement
  with TTestNonPredictiveStrategy
  with LimitedRegressionPredictiveStrategy
  with DefaultHistoryPath
  with BufferedSerialization
  with NoLogger

Adaptive.initialize(config)
\end{lstlisting}

\subsection{Extending the framework}

The framework can be simply extended without actually modifying it by creating custom implementations of some of the traits that are used by the invocation and selection process, and by supplying it to the \inlinecode{Adaptive} initialization using a custom configuration, as described in \ref{subsec:framework_config}.

An overview of the areas that are the simplest and most useful to extend will follow.

\subsubsection{Selection strategies}

A new selection strategy can be created by implementing the \inlinecode{SelectionStrategy} trait and by providing the implementation as either predictive or non-predictive selection strategy. The trait itself is technically a single function which is given a sequence of function run histories and an input descriptor and is supposed to return the run history of the function with best expected performance:

\lstset{style=Scala}
\begin{lstlisting}
def selectOption(records: Seq[RunHistory[TMeasurement]], 
  inputDescriptor: Option[Long]): RunHistory[TMeasurement]
\end{lstlisting}

The strategy will most likely have to work with either a specific evaluation data type, or at least put a constraint on it (by using the \textit{viewability} mechanism, see section \ref{subsec:implicit_args}).

\subsubsection{History storages}

If we wanted to change the way in which the history data are stored, e.g. by filtering some of the results, performing some aggregations to save space, etc., we would need to implement one or both of the following traits - the \inlinecode{RunHistory} trait, that represents the sequence of results for one group and function, and the \inlinecode{HistoryStorage} trait, that manages the \inlinecode{RunHistory} instances for different groups and functions.

The main advantage is that these implementations can work with unspecified evaluation data type, if they are supposed to work just like a storage, or to have a fixed evaluation data type in case when some data aggregations or precomputations are required.

\subsubsection{Evaluation data}

We can change the \inlinecode{TMeasurement} data type and make the framework perform the adaptations according to a different (or more complex) evaluation results. In such a case, we need to implement the \inlinecode{EvaluationProvider} trait, and then add our custom selection strategies that select based on the new evaluation type.
\chapter{Evaluation}

An~example citation: \cite{Andel07}


\section{Artificial algorithm}
\section{Sorting algorithm}
\section{... General algorithm - predicting the big O notation ...}
\section{SQL query selection}
\section{Parsing / serializatin}
\section{Spark}
\subsection{What is Spark}
\subsection{Usage}
\subsection{Environment}
\section{Server selection}

\section{Problems with the practical use of the framework}

\subsection{Selection overhead}
\subsection{Maintainability}

One of the main problems that such a framework is facing in a real-life software system is an engineering problem. The code of today's systems has to be kept working and maintained for several years, sometimes decades. During this period, bugs appear in the system, business requirements change and consequently, it is necessary to perform changes and adaptations in function code.

Maintaining multiple implementations of the same functionality at once brings a lot of issues. The developers have to know exactly how all the implementations work and whenever it's necessary to modify the behavior, be able to perform changes in all of them to achieve the same result. Described process itself is very demanding and has a significant time impact on the development. 

What's more, subtle differences in behavior of the implementations could be unintentionally introduced. These differences might not have visible effects immediately and may appear after several other modifications. At that moment, it will be extremely difficult to locate the problem, because the misbehavior caused by it won't be deterministic thanks to the nature of the selection algorithm.

Using non-deterministic decision tools in general lowers the maintainability of the system.

\subsection{Debugging}

Discovering and fixing bugs in a software system is a key part of its development and maintenance process. It is a very complex and time consuming activity, and its complexity grows considerably for the bugs that have non-deterministic occurrences. Whenever there is a bug in one of the implementations used with the framework, it is inherently non-deterministic, as it has effect only when the specific implementation is selected.

\section{Other tests...?}

\section{Usage from Java}

\section{Usage from Kotlin}

%TODO: Usage from Java & Kotlin

\section{Implementation in other languages}

The framework is implemented in Scala and can be used by any JVM-based languages, even though often without the comfort that provides Scala with its implicits and eta-expansion.

The same functionality, however, could be used on other platform than JVM as well. The framework itself isn't complicated to reproduce and its runtime back-end is transferable to basically any platform. There would be minor complications with immutability of the data structures and expressiveness of Scala in some cases, but nothing show stopping.

The key part of the framework that would cause problems upon reimplementation is the API, which relies heavily on Scala's DSL features (see \ref{sec:dsls}). It requires the following features from the language:

\begin{itemize}
	\item Implicit type conversions
	\item Functions as first-class values
	\item Eta-expansion of methods
	\item Infix operator syntax for methods
	\item Macros to parse and modify the AST upon compilation
\end{itemize}

With a subset of the features in the language, a limited API can be designed. Let's have a look at some language examples and what they offer.

\subsection{Kotlin}

\subsection{Java}

\subsection{C\#}

The C\# language has relatively advanced features and offers function types (\inlinecode{Func<TArg, TRetVal>}) that allow treating the functions as first class values. There is also a feature called \textit{Method groups}, which, in certain situations, lead to an implicit conversion of a method on an object to the corresponding function type - basically the same as eta-expansion.

%TODO: Add examples

The API in C\# could be based on method chaining and be quite similar to the one in Scala, although the method name extraction and the infix fluent-language syntax wouldn't be present.

\subsection{Python}
???

\subsection{C / C++}
\chapter{Related work}

A different approach for the same goal consists in manually selecting an implementation and guaranteeing its performance with unit tests. A special language, SPL (Stochastic Performance Logic), which allows to express assumptions about the performance of multiple functions, was introduced in \cite{bulej_capturing_2012}. The innovative concept is that it treats the performance as a random variable, not as a fixed value, so the conditions are evaluated with certain confidence. It was later used to examine the performance of JDOM framework in \cite{horky_performance_2013}, and became the base of performance unit testing frameworks for C\# (\cite{Trojanek:Thesis:2013}) and Java (\cite{Kotrc:Thesis:2015}).

The development of adaptive systems that can tweak their performance based on the environment is a widely studied topic as well. Systems that are capable of some form of \textit{auto-tunning}, i.e., self reconfiguration based on an analysis of the execution environment (OS, CPU, etc...), are often among the fastest in its category - an example might be \cite{frigo_fftw:_1998},  which is considered to be the fastest non-commercial FFT\footnote{Fast Fourier transform} algorithm in the world. The adaptation in this case is done by recompiling the application code (\cite{frigo_fast_1999}).

An idea of a universal adaptive framework that would not require manual analysis of the environment and work only with observations from actual execution is presented in \cite{bulej_performance_2012}. It describes a system that has generally the same goal as our work - increase performance awareness by designing the systems adaptively. It is focused on large component-based applications where the performance of certain components is tracked. Multiple use case scenarios are presented, e.g. verifying component contract, observing trends, or selecting between multiple interchangeable components based on performance. The SPL is used to make the decisions, i.e., the components are guarded by conditions expressed in the language.

The article does not solve the issue of predicting run time with a certain input based on the performance observed with other inputs. The presented concept is also based on the interconnection between the adaptive framework and the system components. ScalaAdaptive provides a partial implementation of a similar system, more narrowly focused, with higher emphasis on the simplicity and efficiency of performing the single task of selecting faster implementation. In addition, it provides an API that allows easily using the adaptivity in any existing system with no need of architectural changes.

The task of predicting function (or application in general) run time based on the input is a separate problem examined in many different ways. In \cite{wegbreit_mechanical_1975}, a static analysis of LISP programs is performed in order to derive actual complexity. A runtime analysis approach was taken in \cite{goldsmith_measuring_2007}, where basic blocks in the program called \textit{locations} are identified and their performance is measured for a certain input described by a set of \textit{features}. The performance is represented by an execution count in order to keep the measurement deterministic and platform or environment independent. A powerlaw regression is then built to find relations between \textit{feature} values and \textit{locations}. The resulting model can be used to formulate predictions.

Relatively precise predictions of an actual execution can be achieved using the Mantis framework \cite{chun_mantis:_2010}, which is based on instrumenting the code and automatically identifying its \textit{features} (loops, branches, variable values\dots). The \textit{features} are evaluated (branch counts, loop counts\dots) at runtime and the machine learning process is used to select the ones that are important for the overall performance. A regression model is then constructed using the data. One of the possible models is described in \cite{huang_predicting_2010}.

Authors of \cite{smith_predicting_1998} obtain the predictions in two steps - first, they use greedy or genetic algorithm to find similar inputs (jobs, workloads) in the historical run measurement, and then, they generate the prediction using either simple mean or inear regression, both with corresponding confidence interval. The basic approach taken here is quite similar to the ScalaAdaptive, even though neither this, nor the other works referenced in this chapter deal with the problem of real-time prediction upon invocation, where one of the main concerns is minimizing the overhead.

\chapter*{Conclusion}
\addcontentsline{toc}{chapter}{Conclusion}

In this thesis, we designed and implemented a framework that allows a completely new style of performance-aware development by composing functions from interchangeable implementations. An API that allows fluent integration with almost no effort from the developer was introduced. Various statistical methods to identify the most appropriate implementation for given input based on historical performance observation were examined, implemented and compared.

The testing that was carried out identified some potential use-cases for this style of development. Among the more suitable ones were in general longer running functions, where the run time fluctuations were not as significant and where the selection overhead was negligible. We achieved better overall run times of adaptive functions on sequences of various inputs, either for an algorithmic problem (matrix multiplication), or in case of selection between two utility libraries (JSON parsing). On the other hand, optimizing fast-running functions for small inputs does not appear to be a useful application, as the overhead surpasses the potential benefit from adaptation.

In addition, we demonstrated a potential of environment adaptation on the Spark distributed data processing framework, where different configurations and query types achieved better results depending on the execution environment. The Spark itself has a big potential in the adaptivity, as it is being actively developed and many features have experimental character and get iteratively optimized.

Among the main drawbacks of the solution is the fact that the user still has to identify the key feature of the input, which has to be one-dimensional (simple integer) in order to enable input based selection. The prediction models are not perfect, especially in cases where the observed performances of all the functions are very similar. In addition, upon selecting from more than two functions, the system might get stuck in a situation where the selection strategy cannot decide due to a pair of equally good variants and rotates all of them in a round-robin manner.

The majority of current problems and issues can be addressed in the future by extending the framework and adding new functionality. This can be done without actually modifying the framework code due to the modularity and run-time composition.

We believe that the biggest potential of this framework lies in its simplicity and possibility to be used in all kinds of systems without any structural changes. As we do not expect a common developer to implement one functionality multiple times and then combine it using our framework, the main use-case in some larger project would most likely consist of combining multiple libraries, configuration, queries or other simply obtainable entities.

To see the actual benefit that this development approach can bring to a larger system, we would need to practically test it in such an application, i.e. a long running service, and track the performance changes. This kind of test would also expose other problems connected with either the framework, or the whole concept of adaptive development, that could not have been identified in the isolated artificial tests that were part of this work.

\section*{Future work}

\subsection{Selecting from multiple functions}
\label{subsec:problem_selecting_multiple}

As mentioned in \ref{subsec:selecting_multiple_function}, the task of selecting the most suitable function out of a larger set is complicated by its nature and we have taken a simplified approach in this work. In addition, the significance-based selection strategies have to perform multiple tests (or confidence interval constructions) in the process and the probability of an error increases. We can either keep the significance high and lower the strength of the test, which leads to more situations without any function being selected, or let the significance drop while keeping the strength, causing more selection errors.

The potential future extension could either completely redefine the goal of the selection from multiple functions and design a new process, or extend the current solution by determining suitable significance corrections.

\subsubsection{Extending the policy logic and connecting it to selectors}

As already briefly mentioned in section \ref{subsec:policy_improvements}, the policies are currently relatively limited, both in the data they have to base their decision on, and in the results they can produce. It would be interesting to analyze possible extensions while keeping the evaluation overhead as minimal as possible. The policies could give hints to the selection strategies, limit the data that they are working with, etc.

\subsubsection{Custom selection rules based on the SPL}

The \cite{bulej_performance_2012} introduces an extension of the SPL language designed to describe conditions based on the trends of historical performance data of functions. A special selector could be designed to accept the SPL conditions and make decisions according to them. This would allow the framework user to have more control over the selection process. 

\subsubsection{Improving the persistent storage}

The run data in the persistent storage are currently stored in CSV file format, which is not very fast to serialize and deserialize. In addition, the serialization buffering is done in a very basic way, leading to a loss of run data when the application ends with a non-empty buffer. In addition, this solution still has a non-trivial overhead. A potential improvement task might be to select a new format and a method how to flush the buffer upon closing the application, i.e. using JVM shutdown hooks.

\subsubsection{Thread safety and concurrency}

The framework in general is currently not thread safe. It is designed, however, with regard to the possible concurrency, so the modifications should not be dramatic. For the possibility to safely use the framework from a multi-threaded environment, we would need to ensure three basic points:

\begin{enumerate}
	\item Thread-safe initialization and static runtime composition
	\item Thread-safe access to the run history
	\item Thread-safe access to the combined function state (run statistics and current policy)
\end{enumerate}

The first point should not cause any problems. Run histories are already prepared to be made thread-safe by having an API to support immutability (mutating operations return new instances) and an immutable implementation (with all the caches) as well. Because of that, only adding a simple lock on the history updating method should be enough. As to the third point, this is currently the biggest issue, because it has been optimized for performance and the statistics are mutable in general. They could, however, be made into an immutable structure, and replacing the policy and updating the statistics would be atomic operations without any overhead.

\subsubsection{Optimizing run time of the whole system}

Instead of optimizing the run times of separate functions, we could aim for optimizing the whole system. We could, for example, have a set of different data structures that our algorithm will be working with. The data structures would have all the necessary operations implemented and presented using a trait or an abstract class. The framework would generate a factory method that would create one of the implementations, and then track the overall behavior of different methods on the implementations. 

\lstset{style=Scala}
\begin{lstlisting}
val data = createDataStructure(list)
data.getMin()
data.getMax()
data.add(i)
\end{lstlisting}

\subsubsection{Selection strategies supporting multidimensional input descriptors}

As mentioned in section \ref{subsec:input_in_selection}, algorithm complexities are quite commonly functions of multiple features of the input. In order to create selection strategies that would be able to correctly decide in case of such algorithms, we would need to add support for \textit{input descriptors} with more features. The input based strategies would then have to construct multidimensional regression models and analyze the dependency on multiple features.

\subsubsection{Detecting the input features}

The fact that the user has to know which factors of the input might affect the function run time and manually specify the \textit{descriptor function} and the grouping, is quite limiting. Another way of improving the framework might be some kind of analysis that would examine the history measurements and look for correlation between the input feature changes and the run time changes. The main limitation is that the input has to be some structure known to the framework in order to extract the features.

\subsubsection{Significance level analysis}

Most of the strategies used have a significance level parameter for their decisions. The significance gets affected by performing multiple tests or confidence interval constructions during selection between more than two functions. The theory, as described in sections \ref{subsec:simple_linear_regression} and \ref{subsec:t_test_multiple}, would allow us to perform estimations of the impact of multiple function testing on the significance level, which would be quite complicated. 

More interesting future work, however, might be based on experimentally determining the most appropriate corrections for various number of functions based on real cases and real function behavior.

\subsubsection{Implementation in other languages}

The framework is implemented in Scala and can be used to combine functions for any JVM-based languages (see section \ref{subsec:usage_from_java}). Transferring the whole framework to a different language and platform might be desirable in the future. In such a case, the main problem would be the API of the framework, which, in its current design, relies heavily on Scala's DSL features (see section \ref{sec:dsls}). It requires the following features from the language:

\begin{itemize}
	\item Implicit type conversions
	\item Functions as first-class values
	\item Extensibility of the function types
	\item Eta-expansion of methods
	\item Infix operator syntax for methods
	\item Macros to parse and modify the AST upon compilation
\end{itemize}

With a subset of these features in the language, a limited version of the current API could be designed. 

The best approach for a different platform implementation would probably be to analyze the features that the target language offers and the common approach to designing APIs in that language, and to create a new, better suited API.

%%% Bibliography
\include{bibliography}

%%% Figures used in the thesis (consider if this is needed)
\listoffigures

%%% Tables used in the thesis (consider if this is needed)
%%% In mathematical theses, it could be better to move the list of tables to the beginning of the thesis.
\listoftables

%%% Abbreviations used in the thesis, if any, including their explanation
%%% In mathematical theses, it could be better to move the list of abbreviations to the beginning of the thesis.
\chapwithtoc{List of Abbreviations}

\begin{itemize}
	\item \textbf{HTML} - Hyper Text Markup Language
	\item \textbf{API} - Application Programming Interface
	\item \textbf{REST} - Representational State Transfer, a philosophy of web API
	\item \textbf{JSON} - JavaScript Object Notation, a format for data exchange
	\item \textbf{XML} - Extensible Markup Language, a format for data exchange
	\item \textbf{FFT} - Fast Fourier Transform
	\item \textbf{JAR} - Java Archive, a package format to distribute JVM based applications
	\item \textbf{SPL} - Stochastic Performance Logic
	\item \textbf{OS} - Operating System
	\item \textbf{CPU} - Central Processing Unit
	\item \textbf{I/O} - Input / Output
	\item \textbf{IoC} - Inversion of Control, an application development pattern
	\item \textbf{AST} - Abstract Syntax Tree, the result of syntax analysis of a code
	\item \textbf{DSL} - Domain Specific Language
	\item \textbf{SBT} - Scala Build Tools
	\item \textbf{RDD} - Remote Distributed Dataset, a Spark API
\end{itemize}

%%% Attachments to the master thesis, if any. Each attachment must be
%%% referred to at least once from the text of the thesis. Attachments
%%% are numbered.
%%%
%%% The printed version should preferably contain attachments, which can be
%%% read (additional tables and charts, supplementary text, examples of
%%% program output, etc.). The electronic version is more suited for attachments
%%% which will likely be used in an electronic form rather than read (program
%%% source code, data files, interactive charts, etc.). Electronic attachments
%%% should be uploaded to SIS and optionally also included in the thesis on a~CD/DVD.
\chapwithtoc{Attachments}
\label{attachments}
\renewcommand{\thesection}{\Alph{section}}

\section{The enclosed disc contents}
\label{attach:cd}

\section{The Scaladoc documentation}
\label{attach:scaladoc}

\section{Complete list of configuration blocks}
\label{attach:config_blocks}

The following blocks can be used to configure ScalaAdaptive. If there is not a suitable block, a custom configuration implementation can be done.

\subsubsection{Analytics}
\begin{itemize}
	\item \inlinecode{AnalyticsCollection} - turns on the analytics data collection
	\item \inlinecode{NoAnalyticsCollection} - turns off the analytics data collection
\end{itemize}

\subsubsection{Logging}
\begin{itemize}
	\item \inlinecode{ConsoleLogging} - framework logs to the standard output
	\item \inlinecode{FileLogging} - framework logs to a specified file (\textit{logFilePath} argument)
	\item \inlinecode{NoLogging} - framework does not log
\end{itemize}

\subsubsection{Persistence}
\begin{itemize}
	\item \inlinecode{BufferedPersistence} - framework persists the run data in batches of \textit{serializationBufferSize} into a directory (\textit{rootHistoryPath} argument)
	\item \inlinecode{DirectPersistence} - framework persists the run data into a directory (\textit{rootHistoryPath} argument)
	\item \inlinecode{NoPersistence} - framework does not persist the data, global storage is used for persistent setup
\end{itemize}

\subsubsection{Selection strategies}
Sets the corresponding strategy as the main strategy of given type. Note that the \inlinecode{LowRunAwareStrategy} is always used as a wrapper with configurable \textit{lowRunLimit} argument. Some strategy blocks allow setting \textit{alpha} and \textit{averageWindowSize} arguments.
\begin{itemize}
	\item \inlinecode{LinearRegressionInputBasedStrategy}
	\item \inlinecode{LoessInterpolationInputBasedStrategy}
	\item \inlinecode{TTestMeanBasedStrategy}	
	\item \inlinecode{UTestMeanBasedStrategy}
	\item \inlinecode{WindowBoundRegressionInputBasedStrategy}
	\item \inlinecode{WindowBoundTTestInputBasedStrategy}
\end{itemize}

\subsubsection{History}
\begin{itemize}
	\item \inlinecode{CachedGroupHistory} - caches average run times for input descriptors
	\item \inlinecode{CachedRegressionHistory} - caches linear regression model
	\item \inlinecode{CachedStatisticsHistory} - caches statistics (used for T-test)
\end{itemize}

\openright
\end{document}
