\chapter*{Introduction}
\addcontentsline{toc}{chapter}{Introduction}

In modern programming tasks, performance awareness is becoming a first class concern, as we work with growing amounts of data, put higher emphasis on the real-time processing, implement embedded solutions as a part of the IoT and in general, as the requirements demand faster software, but we are reaching the hardware limitations. 

Performance in general is a dynamic, not static feature. Even though we might analyze the algorithm complexity at the design-time, we will always get only an approximation, asymptotic in general, as we work with abstract elementary computational steps. In reality, the programs often behave in a different way, due to problems with CPU cache and memory locality, I/O operations, parallelism and many other factors. The only way how to actually find out which implementation has better performance is to do a thorough performance test covering all the possible input and run in the actual production environment.


 

%TODO: Chapter overview

In the first chapter of this work, the Scala language features that are not common among the current object-oriented programming languages are introduced. Most of them will be mentioned and used in further chapters, so it serves as a brief introduction for a reader that knows basic OOP and functional languages, but does not have a deep knowledge of advanced Scala features.

The second chapter serves as a brief overview of the framework developed as a part of this thesis. More detailed goals are presented and a basic structure of the solution is outlined.

The third chapter is dedicated to the framework API. The reader is guided through the entire process of designing the API, from the requirement analysis and most simple use-cases, through possible drafts and their problems, the options that Scala offers, the actual design, its usage and problems, ending with more advanced scenarios and extensions. The whole chapter is written independently on the framework implementation, is based only on the requirements and a basic notion of the functionality.

The fourth chapter is more theoretically focused. It explains the functionality of the selection part of the framework on an abstract level. The whole process of deciding between multiple functions is explained, and algorithms used within are presented, tested and compared. In addition, more concrete improvements are suggested to improve the selection and to speed up the invocation process.

In the fifth chapter, the whole framework implementation is briefly presented. Concepts from the third and fourth chapters are incorporated into an actual Scala program. A couple of problems are mentioned and solved. Last but not least, the options that the framework user has of customizing and possibly extending the functionality are discussed.

The goal of the seventh chapter is to try out the framework in real-life scenarios and to evaluate benefits that it brings. For that, a couple of different usage scenarios will be used, including basic selection between algorithms, JSON parsing or adapting to quickly changing network environment. A part of the chapter is devoted to the Spark framework for distributed data processing, which offers a great potential for adaptive execution.

In the last, eighth chapter, other work that is related to the problem of performance adaptation, measurement, prediction and similar topics is analyzed.