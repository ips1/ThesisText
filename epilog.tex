\chapter*{Conclusion}
\addcontentsline{toc}{chapter}{Conclusion}

In this thesis, we designed and implemented a framework that allows a completely new style of performance-aware development by composing functions from interchangeable implementations. An API that allows fluent integration with almost no effort from the developer was introduced. Various statistical methods to identify the most appropriate implementation for given input based on historical performance observation were examined, implemented and compared.

The testing that was carried out identified potential use-cases for this style of development. Among the more suitable ones were in general longer running functions, where the run time fluctuations were not as significant and where the selection overhead was negligible. We achieved better overall run times of adaptive functions in artificial scenarios with input adaptation, either in case of an algorithmic problem (matrix multiplication), or in case of selection between two of the most used libraries.

In addition, we demonstrated a potential of environment adaptation on the Spark distributed data processing framework, where various configurations and query types achieved better results in different environments. The Spark itself has a big potential in the adaptivity, as it is being actively developed and many features have experimental character and are being iteratively optimized. On the other hand, the overhead caused by the invocation effectively prevents us from using the adaptation with quick and simple functions.

Among the main drawbacks of the solution is the fact that the user still has to identify the key feature of the input, which has to be one-dimensional (simple integer) in order to enable predictive selection. The prediction models are not perfect, especially in cases where the observed performances of all the functions are very similar. When we are, in addition, selecting from more than two functions, we might get stuck in a situation where the selection strategy cannot decide due to a pair of equally good functions and all of the functions are being used in a round-robin manner.

A majority of the current problems and issues can be addressed in the future by extending the framework and adding new functionality. This can be done without actually modifying the framework code due to the modularity and run-time composition.

We believe that the biggest potential of this framework lies in its simplicity and possibility to be used in any kind of system without any structural changes. As we do not expect a common developer to implement one functionality multiple times and then combine it using our library, the main use-case in some larger project would most likely consist in combining multiple libraries, configuration, queries or other simply obtainable entities.

To see the actual benefit that this development approach can bring to a larger system, we would need to practically test it in such an application, i.e. a long running service, and track the performance changes. This kind of test would also expose other problems connected with either the framework, or the whole concept of adaptive development, that could not have been identified in the isolated artificial tests.