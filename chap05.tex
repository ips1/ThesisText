\chapter{ScalaAdaptive library implementation}
\label{chap:implementation}

\section{Goals of implementation}

The main concerns upon implementing the library were the following:

\begin{itemize}
	\item To separate the API from the selection logic
	\item To make the library as extensible as possible
	\item To keep the library overhead as low as possible
	\item To use the Scala language features to make the code simpler and easier to maintain
\end{itemize}

\subsection{Development approach}

\begin{itemize}
	\item Prefer immutable structures
	\item Prefer composition before inheritance
	\item Use functional approach
	\item In composition, use only trait types
	\item Prefer Optional passing (railway oriented programming) for error handling
\end{itemize}

\subsubsection{Error handling}

The library is designed so that the number of exceptions handled in the code was minimized. The library itself doesn't raise any exceptions in case of errors, and catches most of the exceptions from the libraries within to replace them with a None return value.

This approach is known from the functional programming and takes advantage of monadic operations over the \textit{Option monad}\footnote{In Haskell and other languages known as Maybe monad.}. The return values can be mapped over using the \textit{bind} operator, which allows smooth function chaining and the error propagation through the chain.

\section{Main architecture}
\subsection{API implementation}
\subsubsection{MultiFunctions and FunctionAdaptors}
\subsubsection{FunctionConfig}
\subsection{History storage}

\subsection{Run history storage location}
\label{subsec:storing}

The main concern of the selection mechanism design that is invisible to the user is all the data that will be gathered at runtime and used to select the appropriate variant. Assuming we have a function combined from two different implementations and it gets invoked at various points throughout the execution of the process, we have multiple options where to store the measurements.

All of the options were actually implemented and the API allows the user to chose storage location for every usage of the combiner.

\subsubsection{Local}

The first and most straightforward approach is to store all the measured runtime data inside the function object itself - we have a custom type \inlinecode{FunctionAdaptor} that is hidden to the user and that carries the references to function implementation, so it could hold the measurements for them as well.

The measurement data are specific for every instance of the adaptor and limited to it. This could have some use cases, for example classes that hold immutable data and provide some functionality that is based on the contained data, which is expected to be invoked repeatedly. The run history will then be specific for the data held by the instance and thus perfectly reflect performance in this one case.

There is a limitation connected to this use case - user of the API himself has to make sure that the adaptor instance will survive and be used in every call. Let's consider the following example:

\lstset{style=Scala}
\begin{lstlisting}
def processData(data: List[Int]): Int = 
(impl1 _ or impl2 withStorage Storage.Local)(data)
\end{lstlisting}

In this case, during every invocation of the method \inlinecode{processData}, there will be new local instance of the \inlinecode{FunctionAdaptor} created by the \inlinecode{or} call. This instance will be used for the single call and then thrown away with the measured data. In order for this example to work properly, the adaptor has to be stored away:

\lstset{style=Scala}
\begin{lstlisting}
val impl = impl1 _ or impl2 withStorage Storage.Local
def processData(data: List[Int]): Int = impl(data)
\end{lstlisting}

\subsubsection{Global}

Another option is to store all the measured data globally, in a static area of the memory accessible from all contexts. A unique identifier has to be used to identify every one of the functions and methods combined.

The main advantage of this approach is that the run data are collected and stored from all invocation in the entire process, so there is more data to base the decision process on. This is usually the preferred approach for some business logic or utility methods in long-running services.

\subsubsection{Persistent}

The globally accessible runtime data can be 

This would be most useful (and probably the only means of using the ScalaAdaptive) for short-running tools with more variants of implementing the main logic. If the logic is invoked only once (or only a couple of times) during one invocation, there will never be enough data collected in a single run and it is necessary to persist the run history, saving data from multiple runs.

As the selection algorithms are delivering better results with growing data size, persisting the run history improves the selection mechanism in all the programs.

What can be a little problematic is deciding on when to persist the data. The logical solution to minimize the time spent on writing data is to store the entire run history once at the end of execution of the program. The runtime support, however, doesn't have any method of detecting when the program is about to end - JVM doesn't guarantee finalizers being called and relying on notifications from the user would complicate the API and generally be problematic.

The other approach is to update the persisted history with every run. Its downside is that the I/O operation might have negative impact on the performance if the function is called often and its run is fast. This impact can be limited by buffering the I/O - using the default Java I/O buffering mechanism would require leaving the file open for the entire application run, so custom buffer would be preferred. The buffer could be flushed upon every n-th write.

There are more problems connected with this solution, namely:
\begin{enumerate}
	\item Running multiple instances of the application at once (collisions on the persisted data file) - especially if the file is left open
	\item Changes in the application code (outdated run data, changed identifiers, etc.)
	\item ...
\end{enumerate}
%TODO: More


\subsection{Run history storage identifiers}

In the globally accessible run history data storage, the data have to be assigned to a specific function. A unique identifier that would be derivable from the function objects passed into the \inlinecode{FunctionAdaptor} is required, so that whenever an adapted function is being executed, the runtime can access its history. 

In this case, references can't be used, because the \inlinecode{FunctionAdaptor} objects and the function objects wrapped inside can exist in multiple instances, can be deallocated and reallocated.

\subsection{Type name}

Functions in Scala are instances implementing the \inlinecode{FunctionN} trait. The default implementations are anonymous closure classes that are compiled from lambda expressions. Two different functions originated in two different lambda expressions and thus have two different type names, the ones that were generated and assigned by the compiler. The fully qualified type name can be used as the identifier of a function

Using the type names as unique identifiers is safe and straightforward. A small disadvantage is that they aren't very readable, the compiler uses the name of the type that contained the lambda expression followed by sequential number. 

There is also a subtle danger connected - in case of persisting the run history, the closure classes might get renamed automatically upon recompiling. The compiler usually assigns the closure names sequentially, so this could happen by just inserting another lambda expression into the enclosing class code before the current one. Run history data might even get mixed up as the newly added lambda expression could have the former name of the original closure (by taking its position in the sequence).

If the \inlinecode{FunctionN} trait has a different than default implementation, the type name identification might fail - the trait can be implemented by a single class wrapping other values. In this case, all functions implemented by that class would share the run history.

\subsection{Method name}
\label{subsec:methodnameident}

The most common usage pattern is the one where the functions used with the \inlinecode{or} method are eta-expanded methods (see \ref{subsubsec:apimethods}). The eta-expansion internally replaces the method name with a lambda expression wrapping the method call into a function. Consider the following example:

\lstset{style=Scala}
\begin{lstlisting}
def method(x: Int): Int = ???
def printName[T, R](fun: (T) => R) =
println(fun.getClass.getTypeName)
...
printName(method)
\end{lstlisting}

Line 5 will print out \textit{Test\$\$anonfun\$main\$1}, a name of a closure class generated to encapsulate lambda expression. The closure itself gets compiled and so at runtime, there is no way of finding out, which method was called inside (i.e. which method it was expanded from).

If we, in case of eta-expanded methods, use the type name identifier, we run into a little problem - if a method is used in multiple \inlinecode{or} expressions, a different closure class is generated for each expression where it's eta-expanded. So in the following example, run history data of \inlinecode{method1} wouldn't be shared for runs originating from \inlinecode{combined1} and \inlinecode{combined2}:

\lstset{style=Scala}
\begin{lstlisting}
val combined1 = method1 _ or method2
val combined2 = method1 _ or method3
\end{lstlisting}

More convenient would be to use directly the names of the methods that are being eta-expanded. The names have to be extracted at compile time, using the def macros (see \ref{sec:defmacros}). At the moment of implicit conversion from \inlinecode{FunctionN} to \inlinecode{FunctionAdapterN}, the AST\footnote{Abstract syntax tree.} of the \inlinecode{FunctionN} expression can be examined - if it's a lambda expression containing one method call, the method name can be extracted.

Using method name identifier has some disadvantages as well. Firstly, all method overloads share the same identifier. Additionally, generic type arguments are not included in the identifier either.

\subsection{Custom identifier}

In order to handle the specific cases where type name and method name identifiers don't distinguish correctly between different functions, there is also a possibility of choosing a custom, arbitrary identifier. This has to be triggered specifically by the user in the API and should be used in the cases where:

\begin{enumerate}
	\item User knows that the automatically assigned identifiers won't be sufficient
	\item User wants to replace default type identifier with custom identifier because of readability
\end{enumerate}

\section{Extracting method name from eta-expansion AST}

In order to retrieve the method name to be used as an identifier (as explained in \ref{subsec:methodnameident}), we need to do the following:

\begin{enumerate}
	\item Replace the implicit conversion method from \inlinecode{FunctionN} to \inlinecode{FunctionAdaptorN} by a def macro (see \ref{sec:defmacros})
	\item Inside the macro, analyze the AST, detect eta-expansion method call
	\item If there is a method call:
	\begin{enumerate}
		\item Generate the identifier expression as a \inlinecode{getTypeName} call on the target of the method call, followed by the method name
		\item Generate the conversion code with explicitly specified reference expression
	\end{enumerate}
	\item Otherwise generate the conversion code with implicit reference	
\end{enumerate}

The conversion is done using the \inlinecode{toAdapter()} method with two overloads:
\begin{itemize}
	\item Accepting only the function - implicit reference is used (type name of the closure)
	\item Accepting the function and a custom reference - the reference provided is used
\end{itemize}

\subsection{Eta-expansion AST format}

First step in the macro implementation has to be parsing the input AST and detecting patterns that are generated from eta-expansions by the compiler. Using the \inlinecode{printAst()} macro mentioned in \ref{subsec:buildingast}, the following facts were discovered:

\begin{itemize}
	\item The eta-expansion is already replaced by the equivalent code in AST, so it can't be detected directly.
	\item The result of eta-expansion is a lambda expression (function literal).
\lstset{style=Dump}
\begin{lstlisting}
Function(...)
\end{lstlisting}
	\item The lambda expression is always wrapped in a block, being its return value.
	
\lstset{style=Dump}
\begin{lstlisting}
Block(
  List(...), 
  Function(...))
\end{lstlisting}	
	
	\item If the target of the invocation is either a constant or \textit{this}, it is captured in the lambda expression closure (i.e., the constant or \textit{this} is referenced directly from the function body).
	
	\item If the target of the invocation is a variable or a result of a more complicated expression, it is extracted to the enclosing block, its result is stored in a variable local to the block and then captured in the lambda expression closure. The reason is probably to avoid multiple evaluation of expressions with possible side-efects upon every invocation of the resulting function, and to avoid the target being changed throughout the lifetime of the function.
	
\lstset{style=Dump}
\begin{lstlisting}
Block(
  List(
    ValDef(
      Modifiers(SYNTHETIC), 
      TermName("eta$0$1"), 
      TypeTree(), 
      Apply(
        Select(
          This(
            TypeName("ClassName")), 
          TermName("getClassInstance")), 
        List()))), 
  Function(...))
\end{lstlisting}

	\item The function node contains argument definition and the expression itself, which is a single function application.

\lstset{style=Dump}
\begin{lstlisting}
Function(
  List(
    ValDef(
      Modifiers(PARAM | SYNTHETIC), 
      TermName("arg"), 
      TypeTree(), 
      EmptyTree)), 
  Apply(...))
\end{lstlisting}

	
%	\item The lambda expression (function literal) generated for the eta-expansion contains only the method invocation and the type applications (if the method is generic) - all the other expressions that are part of the eta-expansion are extracted to an outer, enclosing block, their results are stored in variables local to the block and then captured in the lambda expression closure. The reason is probably to avoid multiple evaluation of expressions with possible side-efects upon every invocation of the resulting function.
\end{itemize}

This has a few consequences for our case. We need to generate our conversion and method retrieval code into the block return value, because we need to be able to access the invocation targets that can be defined in the block itself. The target and the method name will always be in the single Apply node representing the function body.

If the method doesn't accept any type arguments, the tree is quite simple:

\lstset{style=Dump}
\begin{lstlisting}
Apply(
  Select(
    ...invocation target expression..., 
    TermName("methodName")), 
  List(...function arguments...))
\end{lstlisting}

Where the \textit{invocation target expression} can have multiple forms based on the original expression, and can depend on the enclosing block variables. It isn't, however, important for us, as we can work with the expression as whole. The function arguments are not needed either.

If the method is generic, the type arguments need to be applied in order to convert it to a function (which can't be generic). In this case, the tree gets a little more complicated:

\lstset{style=Dump}
\begin{lstlisting}
Apply(
  TypeApply(
    Select(
      ...invocation target expression..., 
      TermName("genericMethod")), 
    List(...type arguments...))), 
  List(...function arguments...))
\end{lstlisting}

The method call is wrapped in a TypeApply node before being invoked using the Apply node. The TypeApply node can in our case be ignored.

And the most complicated case we can encounter is when the method has some implicit arguments as well:

\lstset{style=Dump}
\begin{lstlisting}
Apply(
  Apply(
    TypeApply(
      Select(
        ...invocation target expression...,
        TermName("genericMethodImplicit")), 
      List(...type arguments...)), 
    List(...function arguments...)), 
  List(...implicit arguments...))
\end{lstlisting}

One more Apply node is added to the topmost level - the implicit arguments are applied after applying the actual function arguments. Their definition contains another nested block and lambda expression, but again, it is not necessary for our case. We just need to extract the invocation target and the method name.

\subsection{Generating the conversion}

Supposing we have the invocation target expression and the method name, we need to create the identifier string expression that will be used in the manual \inlinecode{toAdapter} invocation.

In order to extract the fully qualified name of the method call target, we need to generate the following expression:

\lstset{style=Scala}
\begin{lstlisting}
invocationTarget.getClass.getTypeName + ".methodName"
\end{lstlisting}

The AST representing this expression has to be wrapped in the \inlinecode{MethodNameReference} construction application (it is a case class with automatically generated \inlinecode{apply()} method) to get the resulting reference.

In all the cases, the \inlinecode{toAdapter} method has to be generated to replace the macro function. The function literal from the original AST has to be passed in as the first argument, and optionally, the second argument can be provided if the extraction of the method name was successful.

The original simplified tree looks like this:

\begin{forest}
	[Block
	  [Statements
	    [ValDef]
	  ]
	  [Function]
	]
\end{forest}

We need to transform the block, keeping the definition in the statement part, but wrapping the \inlinecode{Function} into the \inlinecode{toAdapter} call.
	
\begin{forest}
	[Block
	[Statements
	[ValDef]
	]
	[Apply
	  [toAdaptor]
	  [ArgumentList
	    [Function]
	    [ExtractedReferenceExpression]
	  ]
	]
	]
\end{forest}

\subsection{Extracting method overloads}

The approach that was described so far has one small issue - it doesn't recognize function overloads, so all the overloads share the same identifier.

 It wouldn't be difficult to extract the actual number of arguments that the method is being invoked with and include it in the reference. When attempting to extract the argument types of the lambda expression, we encounter a major problem - the function literal is generated without argument type specification, the TypeTree is empty:

\lstset{style=Dump}
\begin{lstlisting}
ValDef(
  Modifiers(PARAM | SYNTHETIC), 
  TermName("i"), 
  TypeTree(), 
  EmptyTree)
\end{lstlisting}

This can be done and is quite similar to the following piece of code:

\lstset{style=Scala}
\begin{lstlisting}
val function: (Int) => Int = { i => math.abs(i) + 1 }
\end{lstlisting}

In this case, the lambda expression doesn't have the type of its arguments specified either, because the compiler will infer it from the context in which the expression is used, in this case, from the type specifier of the variable.

The compiler is able to infer the data type from the usage of the argument as well:
\lstset{style=Scala}
\begin{lstlisting}
def method(i: Int): Int = ???
val function = { i => method(i) }
\end{lstlisting}

The previous piece of code is correctly compiled by the Scala compiler, although some IDEs (namely IntelliJ IDEA 2016.3.4) aren't able to infer the type and mark the code as incorrect.

Note that the eta-expansion is guided by the same rules, so whenever expanding a method without overloads, the compiler infers the types by itself:
\lstset{style=Scala}
\begin{lstlisting}
def method(i: Int): Int = ???
val function = method _
\end{lstlisting}

Upon expansion of a method with overloads, the resulting function type has to be provided and the inference flow goes in the other direction:
\lstset{style=Scala}
\begin{lstlisting}
def method(i: Int): Int = ???
def method(s: String): Int = ???
val function: (String) => Int = method _
\end{lstlisting}

As a consequence, at the time of syntax analysis, the types aren't inferred yet and there is no way for us to find out which method overload is being called.

The method overloads have to share the same identifier, but in most typical cases, this isn't a problem. Otherwise, it can be solved by using custom identifiers.
	
\section{Macros in API}

%TODO: implicit conversions as macros, other problems arising

\subsubsection{Serialization}
\subsection{Run selectors}
\subsubsection{Discrete}
\subsubsection{Continuous}
\subsection{Policies}
\subsection{Configuration}

\section{Extending the functionality}
