\chapter{Adaptive development API}
\label{chapter:api}

API (Application Programming Interface) is a key component of every framework or library. It defines how a stand-alone piece of code (a function, class, module, or even an entire running application) interacts with its environment. This includes the possible calls or requests, format of the data that are passed in as arguments and the data that are received as the result of the action. In our case, the API will be used to access functionality of our library from the user's own code.

One of the key tasks of this thesis is to research the potential possibilities of adaptive development in the user code using the API of the adaptation framework, to design it to be as fluent and expressive as possible using the Scala programming language features described in chapter \ref{chap:scala} and to analyze the problems and limitations that it has. 
Some key design decisions about the functionality of the entire framework that are tightly connected to the API design have to be discussed as well.

\section{Design of the API}

\subsection{Basic API requirements}

The framework itself should not require much interaction from the programmer. The typical use-case of our API could be described in the following way:

\blockquote{
User has a class with two methods, \inlinecode{method1()} and  \inlinecode{method2()}, and uses the framework simple API to create a third method in the same class, \inlinecode{method()}, that adaptively uses one of the other two methods. If he wants to change the behavior of the adaptation (method selection) process, he can do it directly in the definition of the \inlinecode{method()}.
}

The important point is that we want the result of combining the methods to behave exactly like a normal function or method. The caller should not know and should not need to know that he is using an object that is somehow special. This allows introducing adaptivity to existing project just by changing the method implementation, and no new dependency will be introduced in the majority of the code.

We can make a list of the \textbf{basic functional requirements}:
\begin{enumerate}
	\item Create a method by combining two or more different methods (stating that they can be called interchangeably)
	\item Define some basic adaptation behavior for the method
	\item Make some more complex configuration changes for the whole framework
\end{enumerate}

\subsection{Possible API drafts}
\label{subsec:api_drafts}

The core part of the whole API and the mechanism that will determine the usability of the framework is the method combination definition. JVM classes in general have the limitation that once they are loaded by a \textit{classloader}, they cannot be modified, and neither do their methods. Generating new methods at runtime is therefore extremely difficult. An alternative to that would be to create functions instead, which behave like normal Java objects (see section \ref{sec:metandfun}). From the point of view of the potential caller, a function can be used almost identically as a method. 

This leaves us with two basic approaches to the API.

\subsubsection{API based on methods}

This variant would have to rely at least partially on compile-time actions and would have to work with annotations (in case we do not want to create custom language extensions).

The combination definition could work roughly in the following way:
\lstset{style=Scala}
\begin{lstlisting}
@ClassWithCombinations
class Functions {
  @CombineInto("combinedMethod")
  def method1(arg: Int) = ???
  
  @CombineInto("combinedMethod")
  def method2(arg: Int) = ???
}
\end{lstlisting}

There are multiple ways of implementing such an API. We are going to discuss the following two:

\begin{enumerate}
	\item Using Scala compile-time macros
	\item Using aspect weaving
\end{enumerate}

Scala macros would be able to modify the AST of the \inlinecode{Functions} class to contain three methods (\inlinecode{method1}, \inlinecode{method2} and \inlinecode{combinedMethod}) with the same signature. The typecheck would be done at compile time in the macro execution. Notice a few problems and limitation that this solution has:

\begin{itemize}
	\item The \inlinecode{@ClassWithCombinations} annotation would be necessary for each class that would contain the combined function, because the AST modification of the class can be done only when processing an annotation of the entire class
	\item Consequentially, it would be possible to combine only methods within one class
	\item The IDEs do not process the macros upon code inspections and code highlighting, and would mark the combined methods as non existing, which would be extremely inconvenient (this could potentially be fixed by adding a dummy implementation of the combined method, which would impose more work on the user)
	\item The string-based name of the combined method is error prone - methods could be excluded from the combination due to typos in the name, or an existing method could be overwritten by mistake
\end{itemize}

The second option of using an aspect weaver\footnote{For more information about JVM-based language aspect weaving see e.g. the AspectJ project \cite{here_aspectj_nodate}.} would lead to a very similar API with different limitations. The weaver does not allow us to modify the class and to generate a method, but can intercept calls of annotated methods using an annotation pointcut. We could either intercept all calls on the methods combined into the same group and use the selection mechanism every time (which would not allow the user to invoke any of the combined methods without the selection process), or have a dummy method with empty implementation and intercept its calls. Looking for other methods with annotation from the pointcut would require us to use reflection.

\subsubsection{API based on functions}

This option would work with functions as first-class values (and the objects representing them) entirely at runtime\footnote{With a small exception - a compile-time macro is actually used to fetch the method names, as will be explained in \ref{sec:extracting_method_name}}. It might look the following way:

\lstset{style=Scala}
\begin{lstlisting}
val function1 = ???
val function2 = ???
val combinedFunction = /* Expression to combine function1 and function2 */
val result = combinedFunction(data)
\end{lstlisting}

The \inlinecode{combinedFunction} would be an object extending the \inlinecode{Function1} trait (see \ref{subsec:functiontypes}) and thus usable as a normal function.

The main inconvenience of the design is that it works only with functions. Existing methods can be treated as functions using \textit{eta-expansion} (see section \ref{subsec:etaexpansion}), so methods can be combined as well with almost no effort. The other way around, i.e. creating an adaptive method, is a little more complicated. Both will be discussed later.

Apart from these complications, this approach has a lot of advantages. It can be used at runtime in any context to create a temporary function. The type checking and IDE tooling work on it, it can be fluent and flexible. And, in general, this approach fits the functional concept of Scala.

The function combination expression technically has to be a higher-order function (or method). We would like it to conform to the functional Scala design patterns. As an inspiration, lets have a look at a different case where we generate one function from two different ones -- function composition. This can be done in Scala using \inlinecode{andThen} method defined on the function type traits (see \ref{subsec:functiontypes}):

\lstset{style=Scala}
\begin{lstlisting}
val timesTwo = (x: Int) => { x * 2 }
val plusOne = (x: Int) => { x + 1 }
val timesTwoPlusOne = timesTwo.andThen(plusOne)
\end{lstlisting}

We can achieve even more readable and natural looking code if we take advantage of the infix operator syntax (see \ref{subsec:infixops}):

\lstset{style=Scala}
\begin{lstlisting}
val timesTwoPlusOne = timesTwo andThen plusOne
\end{lstlisting}

Chaining of the methods / operators is possible as well:

\lstset{style=Scala}
\begin{lstlisting}
val manyOperations = timesTwo andThen plusOne andThen plusOne andThen timesTwo
\end{lstlisting}

Inspired by this readable and very simple syntax to compose functions which relies only on the basic syntactic features of Scala, we can try to design the combining expression exactly in the same way:

\lstset{style=Scala}
\begin{lstlisting}
val combinedFunction = function1 or function2
\end{lstlisting}

\subsection{Functional API requirements}

From the two basic drafts presented in section \ref{subsec:api_drafts}, the function-based API will be implemented in this work. The reason is primarily a more fluent integration into the Scala programming style and functional development in general. Based on this decision, we can make the basic outlines of the ScalaAdaptive API to be able to examine its usage patterns and possibilities.

The \inlinecode{or} method, in order to be used as an infix operator, has to be defined on the type that represents the first argument, in this case a function. More specifically, all of the function types mentioned in \ref{subsec:functiontypes}. Because Scala does not support extension methods, we need to introduce a new custom type for functions containing this behavior, and an implicit conversion from the function traits. We will use the same type as for the resulting combined function -- an \inlinecode{AdaptiveFunctionN} type\footnote{Actually a set of types, one for each $N$, in this thesis only from $0$ to $5$}. The following is required:

\begin{itemize}
	\item \inlinecode{FunctionN} is implicitly convertible into \inlinecode{AdaptiveFunctionN}, creating a default with just one implementation
	\item \inlinecode{AdaptiveFunctionN} has an operator \inlinecode{or} that will combine it with a different \inlinecode{AdaptiveFunctionN}, creating a new object with concatenation of the implementations
	\item \inlinecode{AdaptiveFunctionN} is a subtype of \inlinecode{FunctionN} and performs the implementation selection and invocation upon calling \inlinecode{apply()}
\end{itemize}

\subsection{Implementation}
\label{subsec:api_implementation}

The type \inlinecode{AdaptiveFunctionN}, which will be the main part of the interface for the user, will be just a trait hiding a concrete implementation. The user should never come in contact with the actual class, as we do not want him to create instances of it manually or to interact with its API. The implementation itself should wrap a list of functions from which to choose, along with some basic configuration, which will be discussed later.

There are two ways for the user how to get a new \inlinecode{AdaptiveFunctionN} instance:

\begin{itemize}
	\item By using the implicit conversion function
	\lstset{style=Scala}
	\begin{lstlisting}
implicit def toAdaptiveFunction1[T1, R](fun: (T1) => R): AdaptiveFunction1[T1, R]
\end{lstlisting}
	\item By calling the \inlinecode{or} method defined on \inlinecode{AdaptiveFunctionN}
		\lstset{style=Scala}
	\begin{lstlisting}
def or(fun: (T1) => R): AdaptiveFunction1[T1, R]
	\end{lstlisting}
\end{itemize}

The implicit conversions have to be defined for all the function types separately as publicly accessible methods on an object\footnote{A Scala concept for singleton.}. For this purpose, a special object \inlinecode{Implicits} was introduced - it can be imported into any other Scala source file and the implicit conversions become active in the scope (for more details see \ref{sec:implicits}).

The conversion of \inlinecode{FunctionN} to \inlinecode{AdaptiveFunctionN} should check every time, whether the \inlinecode{FunctionN} instance is not already a \inlinecode{AdaptiveFunctionN} instance as well, and if so, just cast it instead of creating a new wrapper. The same should be done by the \inlinecode{or} function, which can just manually call the conversion on its argument.

\section{Usage of the API}
\label{sec:api_usage}

After having designed the basic look of our API, we can examine the usage patterns and go through strengths and weaknesses of the solution.

\subsection{Basic use cases}

\subsubsection{Combining two functions}

The most simple and straightforward use-case is to combine two functions previously stored in variables. We use the implicit type conversion and the \inlinecode{or} method.

\lstset{style=Scala}
\begin{lstlisting}
import scalaadaptive.api.Implicits._
val combinedFunc = function1 or function2
\end{lstlisting}

\textit{Note that the import of the \inlinecode{Implicits} object is necessary every time and will be omitted in the code examples from now on.}

We suppose that \inlinecode{function1} and \inlinecode{function2} have the same type \inlinecode{FunctionN} with the same type arguments. The \inlinecode{combinedFunc} variable will have a type of \inlinecode{AdaptiveFunctionN}. When exposing the resulting function in a public API, it might be better to enforce the \inlinecode{FunctionN} type for the variable or attribute:

\lstset{style=Scala}
\begin{lstlisting}
val combinedFunc: (Int) => Int = function1 or function2
\end{lstlisting}

This will ensure that the \inlinecode{AdaptiveFunctionN} specific API will remain hidden unless having the \inlinecode{Implicits} explicitly imported again.

\subsubsection{Combining multiple functions}

The \inlinecode{or} calls can be chained and used to combine more than two functions in one expression.

\lstset{style=Scala}
\begin{lstlisting}
val combinedFunc = function1 or function2 or function3 or function4
\end{lstlisting}

\subsubsection{Combining function twice}

A combined function can be combined again, no matter whether treated as \inlinecode{FunctionN} or \inlinecode{AdaptiveFunctionN} in the meantime. The resulting combination will be the same as if the combination was done in one expression\footnote{Actually, the \inlinecode{or} call chain is exactly the same case, the \inlinecode{AdaptiveFunctionN} instances are created one by one.}.

\lstset{style=Scala}
\begin{lstlisting}
val combinedFunc1: (T) => R = function1 or function2
...
combinedFunct1(...)
...
val combinedFunc2 = combinedFunc1 or function3 or function4
\end{lstlisting}

\subsubsection{Combining lambda expressions}

Just like named function variables, lambda expressions can be combined directly.

\lstset{style=Scala}
\begin{lstlisting}
val combinedFunc = { () => obj1.call(...) } or { () => obj2.call(...) }
\end{lstlisting}

\subsubsection{Combining methods}
\label{subsubsec:apimethods}

This is expected to be the most common use-case -- we have multiple methods and we want to combine them into a single combined function. As mentioned in \ref{subsec:etaexpansion}, the methods can be easily converted in functions, but unfortunately, in case of the implicit typecast and then a method call, the expansion is not done automatically on the first method. The second method, which is an argument, gets expanded.

\lstset{style=Scala}
\begin{lstlisting}
val combinedFunc = obj.method1 _ or obj.method2
\end{lstlisting}

Note that if the \inlinecode{or} method accepted the \inlinecode{FunctionN} argument and relied on the implicit conversion of the argument as well (just like the call target does), its implicit \textit{eta-expansion} would get blocked and an operator \inlinecode{\_} would be needed as well:

\lstset{style=Scala}
\begin{lstlisting}
val combinedFunc = obj.method1 _ or obj.method2 _
\end{lstlisting}

\subsubsection{Creating a method from combined function}
\label{subsubsec:method_from_combined_func}

Even though functions are so common in Scala, there might be situations where we would need to create a method from our combined function. Thanks to the simplified syntax that allows us to define the method body using a one-line expression, quite an idiomatic way to create this combined method exists:

\lstset{style=Scala}
\begin{lstlisting}
def combinedMethod(arg: T) = (function1 or function2)(arg)
\end{lstlisting}

The method body consists of creating the combined function object and then immediately applying it. Although this does look elegant, it has a major problem - the \inlinecode{AdaptiveFunctionN} instance is created upon every invocation, which leads to an unnecessary overhead and to loss of all the data locally stored in the instance. In addition, there is no way to access the \inlinecode{AdaptiveFunctionN} internal API.

The preferred, a little less convenient way is to create a private field containing the combined function initialized in constructor and to delegate the method calls to it\footnote{It can be thought of as an analogy to the concept of \textit{property} and its \textit{backing field}.}.
 
 \lstset{style=Scala}
 \begin{lstlisting}
 private val combinedInner = function1 or function2
 def combinedMethod(arg: T) = combinedInner(arg)
 \end{lstlisting}

\subsection{Covariance and contravariance}
\label{subsec:covariance_contravariance}

So far, all mentioned usages of the \inlinecode{or} method were limited to functions with the same signatures. Quite common case, however, might be combining multiple functions with slightly different argument and return value types, typically one being a specialized version of the other, for example:

\lstset{style=Scala}
\begin{lstlisting}
def sort1(data: Iterable[Int]): Iterable[Int] = ???
def sort2(data: List[Int]): List[Int] = ???
\end{lstlisting}

We will examine how this can be done on a simplified case with more combinations:

\lstset{style=Scala}
\begin{lstlisting}
val fun1: (Any) => String = ???
val fun2: (String) => Any = ???
val fun3: (String) => String = ???

val fun4 = fun3 or fun2
val fun5 = fun2 or fun3
val fun6 = fun3 or fun1
val fun7 = fun1 or fun3
\end{lstlisting}

We receive compilation error on lines 5 and 8, in the definition of \inlinecode{fun4} and \inlinecode{fun7}. Lines 6 and 7 compile correctly. These are the cases where either:

\begin{itemize}
	\item Return type of the function passed in as an argument is a subtype of the return type of the target function
	\item Argument type of the function passed in as an argument is a supertype of the argument type of the target function
\end{itemize}

The function types in Scala are defined in the following way:
\lstset{style=Scala}
\begin{lstlisting}
trait Function1[-T1, +R] extends AnyRef
\end{lstlisting}

The type arguments representing the function arguments are defined as contravariant and the type argument representing the return value of the function is defined as covariant. This leads to \inlinecode{(String) => String} being a subtype of \inlinecode{(String) => Any} and to \inlinecode{(Any) => String} being a subtype of \inlinecode{(String) => String}. So the \inlinecode{or} method is working flawlessly for the \inlinecode{fun5} and \inlinecode{fun6}, as implicit conversions from subtype to supertype exist for the functions.

In general, the target of the \inlinecode{or} call (i.e. the \inlinecode{AdaptiveFunctionN} instance created by the implicit conversion) determines the type of the resulting function, and only its subtypes can be passed in as arguments. This can solve majority of the cases where it would be enough to put the more general function on the left side of the \inlinecode{or} method. In the case where we wanted the combined function to have more general signature, we can perform the typecast on the first function passed in:

\lstset{style=Scala}
\begin{lstlisting}
def sort1(data: Iterable[Int]): Iterable[Int] = ???
def sort2(data: List[Int]): List[Int] = ???
val sortCombined = (sort2 _).asInstanceOf[(List[Int]) => Iterable[Int]] or sort1
\end{lstlisting}

Potentially, this problem could be solved by having the \inlinecode{or} function generic:
\lstset{style=Scala}
\begin{lstlisting}
def or[J1 <: T1, S >: R](fun: (J1) => S): AdaptiveFunction1[J1, S]
\end{lstlisting}
This, however, leads to problems when being invoked as a macro, so this solution was not used in the final version.

\subsection{Generic methods}
\label{subsec:generics}

%TODO: Solution suggestion - OOP approach, genericity in classes

%TODO: How to solve this issue?

Functions have one major limitation in Scala - they can't have any generic arguments. The signature of a function always has all the argument and return value types specified at compile-time. If we perform the \textit{eta-expansion} on a generic method, we have to specify the type arguments. Otherwise, they will be fixed as the most general achievable within the limits of the covariance and contravariance of the function type (see \ref{subsec:covariance_contravariance}).

In some cases, this might make the function impossible to call:
\lstset{style=Scala}
\begin{lstlisting}
def makeTuple[A, B](a: A, b: B): (A, B) = (a, b)
val fun = makeTuple _
// fun: (Nothing, Nothing) => (Nothing, Nothing)
\end{lstlisting}

In other cases, it might lead just to losing the compile-time type check:
\lstset{style=Scala}
\begin{lstlisting}
def defaultCount[A](list: List[A], item: A): Int = 
  list.count(_ == item)
val fun = defaultCount _
// fun: (List[Any], Any) => Int
\end{lstlisting}

As the \inlinecode{or} method performs combination of functions into another function, the \textit{eta-expansion} has to be done at the point of combination, and the arguments should be specified at that point to prevent the described effects. This effectively prevents us from achieving genericity of the combined function in the same way as with methods. 

We can, however, take advantage of the fact that the type arguments being explicitly set in the process of eta-expansion can be generic type parameters of an enclosing structure (generic class or generic method). With this approach, there are two possible patterns for the workaround:

\begin{enumerate}
		\item Defining a generic method by creating the combination and then calling it immediately:
	\lstset{style=Scala}
	\begin{lstlisting}
def count[A](list: List[A], item: A) = (defaultCount[A] _ or customCount[A])(list, item)
	\end{lstlisting}
	This approach was already mentioned in \ref{subsubsec:method_from_combined_func} and is discouraged from as it leads to repeated creation of the \inlinecode{AdaptiveFunctionN} implementation.
	\item Defining a function field inside a generic class:
	\lstset{style=Scala}
	\begin{lstlisting}
def defaultCount[A](list: List[A], item: A) = list.count(_ == item)
def customCount[A](list: List[A], item: A) = list.filter(_ == item).map((i) => 1).sum

class ListTools[A] {
  val count = defaultCount[A] _ or customCount[A]
}
	\end{lstlisting}
\end{enumerate}

\subsection{Implicit arguments}

As with a lot of other features, the implicit arguments (described in detail in section \ref{subsec:implicit_args}) are supported only by methods, not by functions. Therefore, all of them have to be resolved and fixed when performing \textit{eta-expansion}, along with all the type arguments. 

Lets consider the case where the implicit arguments depend on the value of a type argument. The implicit arguments can be provided manually, or using an implicit variable from the scope where they are being expanded. If the type arguments are being fixed to concrete types, we usually have the implicit implementations at our disposal:

\lstset{style=Scala}
\begin{lstlisting}
def radixSort(list: List[Int]): List[Int] = ???
def bubbleSort[A <% Ordered[A]](list: List[A]): List[A] = ???

val sort = radixSort _ or bubbleSort[Int]
\end{lstlisting}

The implicit argument might, however, often be used to set requirements about the type argument (in the sense of type classes from other languages), like in this example:
\lstset{style=Scala}
\begin{lstlisting}
def sort[A](list: List[A])(implicit ord: A => Ordered[A]): List[A] = ???
\end{lstlisting}

As explained in \ref{subsec:implicit_args}, the goal is to propagate the implicit arguments through several different scopes to the point where the type argument is fixed and the implicit value can be provided. The solution from section \ref{subsec:generics} to expand the type argument to a type argument of an enclosing scope can be extended to the implicit arguments, as the principle is exactly the same -- the implicit argument can just be repeated in the method scope:

\lstset{style=Scala}
\begin{lstlisting}
def sort[A](list: List[A])(implicit ord: A => Ordered[A]): List[A] = (sort1[A] _ or sort2[A])(list)
\end{lstlisting}

Or, preferably, at the level of classes:

\lstset{style=Scala}
\begin{lstlisting}
class Sorter[A](implicit ord: A => Ordered[A]) {
  val sort = sort1[A] _ or sort2[A]
}
\end{lstlisting}

\subsection{Extending traits}
Traits in the role of interfaces represent key element of object-oriented programming approach. Having a combined function exposed through a trait to the rest of the system is expected to be one of the most common usage patterns, as it would allow replacing the combined function with a fixed implementation and vice versa.

A trait in Scala can define both methods and properties. In case of properties, the implementations will override the getter and setter of the property (which can be automatically generated). This allows us to define functions as parts of traits as well, and the Scala syntax will make them almost unrecognizable from methods in the trait user's point of view.

\lstset{style=Scala}
\begin{lstlisting}
trait TestTrait {
  def testMethod(arg: List[Int]): List[String]
  val testFunction: (List[Int]) => List[String]
}
\end{lstlisting}

If we design our own traits for the use in the application, we can intentionally expose the logic using only functions. In that case, implementing the trait using combined functions does not represent any problem. In case that we need to use an existing trait that contains methods, we have to implement them and delegate the call to the combined function using the techniques described in section \ref{subsubsec:method_from_combined_func}.

\lstset{style=Scala}
\begin{lstlisting}
class TestImpl extends TestTrait {
  override val testFunction: (List[Int]) => List[String] = impl1 _ or impl2
  override def testMethod(arg: List[Int]): List[String] = testFunction(arg)
}
\end{lstlisting}

%TODO: summary of the API approach, definition of key terms and points in adaptive lifecycle - method definition, function / closure creation (eta expansion), function combination, 

%TODO: Evaluation, advantages, disadvantages, generation

\subsection{Delayed measuring}
\label{subsec:delayed_measuring}

In some specific cases it might be useful to delay the measurement, or, in general, to measure invocation of a different function than the one that has multiple implementation, as they can affect something else than their own runtime. Some example use cases might be:

\begin{itemize}
	\item Configuration, or generation of configuration
	\item Expression tree building
	\item Query building
	\item Method chain building with lazy evaluation
\end{itemize}

It is obvious that our API, which was designed to remain as simple as possible, does not support this case. We need to slightly extend it.

\subsubsection{Decision context}

Supposing we already have a mechanism to mark the actual function to measure, we encounter another problem. Making a decision when executing the function with multiple implementations automatically generates a context of the decision -- all measured function invocations that were affected by the decision. When measuring the invocation, the framework needs to know to which context it belongs to in order to be able to assign the measured data to the run history of the corresponding function.

The contexts can interleave and we cannot match the decision with the invocation by time or location. It is impossible to decide which invocation belongs to which decision automatically - we need support from the user. He will have to explicitly state the context whenever invoking the measured function. By doing this, he will also pinpoint the moment of decision that affected the invocation.

Practically, we need to introduce a special object - the \textit{invocation token} - which will represent the context. The \inlinecode{AdaptiveFunctionN} instances will have a special, modified version of the apply method, which will not trigger the measure of the selected function, but will generate an invocation token that will be used to measure future function runs. We chose to make it accessible using the \inlinecode{\textasciicircum()} operator, to simulate the fluent application.

\lstset{style=Scala}
\begin{lstlisting}
val getConfig = getFastConfig _ or getSlowConfig
val (config, measure) = getConfig^()
...
measure(() => run(config))
\end{lstlisting}

This approach has one main disadvantage compared to the rest of the API - it breaks the transparency. The user has to know that he is working with the \inlinecode{AdaptiveFunctionN}, not just normal function. His assistance is, however, required by the nature of the problem.


\subsection{Usage from Java and other JVM based languages}
\label{subsec:usage_from_java}

Even though Scala has a very high level of possible interoperability with Java and in general, Scala classes and methods can be called directly from Java code (being just a classes and methods in bytecode), the direct usage of described API from Java is not possible. The reason is that it depends heavily on the syntactic features that Scala provides, including implicit conversions and compile-time macros.

There is, however, a very simple way how to utilize the method combining in a Java or any JVM based language project. Supposing we have multiple methods in Java that are interchangeable and that we want to adaptively combine, we can introduce a simple Scala wrapper class. This wrapper will use the ScalaAdaptive framework to create an adaptive function with said Java methods as the implementation options. The function should be exposed through a method, like described in section \ref{subsubsec:method_from_combined_func}, as Java does not have the concept of functions and the invocation would have to be done using the \inlinecode{apply} method call.

%TODO: Use cases - configuration generation, expression tree building, query building...
